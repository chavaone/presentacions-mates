\chapter{Números reales}

\section{Conjuntos numéricos}

\Exercicio Clasifica los siguientes números en naturales, enteros, racionales o irracionales:

\begin{enumerate}[topsep=0pt]

	\item 25,37

	\item -6/17

	\item 2/5

	\item $-\sqrt{12}$

	\item $ \pi	$

	\item -5

\end{enumerate}


\section{Valor Absoluto}

\Exercicio Desarrolla el valor de las siguientes expresiones:

\begin{enumerate}[topsep=0pt]

	\item $2x - 3 + |2x-3|$

	\item $|x+2| + |x+3|$

	\item $x +|x-3| + |x+5|$
	\item $2x + |x+5| - |x-1|$
	\item $|x-2| + |x+1| - 3$
	\item $|x-4| - |x-3| + 4x$
\end{enumerate}


\section{Intervalos y Entornos}

\Exercicio Representa como intervalos los siguientes conjuntos de números:

\begin{enumerate}[topsep=0pt]

	\item $\{x \in \mathbb{R} / -1 \le x < 5  \} $

	\item $\{x \in \mathbb{R} / -1 \ge x > -5 \} $

	\item $\{x \in \mathbb{R} / -3 < x  \} $

	\item $\{x \in \mathbb{R} / 3 > x \} $

	\item $\{x \in \mathbb{R} / -1 \le x \le 0  \} $
	\item $\{x \in \mathbb{R} / 8 \le x < 10  \} $
	\item $\{x \in \mathbb{R} / -10 < x \le 3  \} $
	\item $\{x \in \mathbb{R} / 3 \le x < 9  \} $
	\item $\{x \in \mathbb{R} / -2 > x \ge -30  \} $
\end{enumerate}


\Exercicio Dados $A=(2,4)$, $B=(-2,4)$ y $C=(-3, \infty)$, calcula:

\begin{enumerate}[topsep=0pt]

	\item $A \cup B \cup C$

	\item $A \cap B \cap C$

	\item $(A \cap B) \cup C$

	\item $(A \cup B) \cap C$

\end{enumerate}


\Exercicio Expresa como entornos los siguientes intervalos:
\begin{enumerate}[topsep=0pt]
	\item $(-5, 2)$
	\item $[-10, 10]$
	\item $[-3, 5]$
	\item $(2,8)$
	\item $(-10, 20)$
	\item $[3,17]$
\end{enumerate}

\Exercicio Expresa como entornos los siguientes conjuntos de números:

\begin{enumerate}[topsep=0pt]

	\item $\{x \in \mathbb{R} / |x-2| < 5 \} $

	\item $\{x \in \mathbb{R} / |x+2| \le 3 \} $

	\item $\{x \in \mathbb{R} / |x-5| < 2 \} $

	\item $\{x \in \mathbb{R} / |x-1| < 6 \} $
	\item $\{x \in \mathbb{R} / |x-9| \le 1 \} $
	\item $\{x \in \mathbb{R} / |x-24| < 5 \} $
\end{enumerate}


\section{Aproximación y Errores}

\Exercicio Aproxima por redondeo o truncamiento los siguientes números como se indica:

\begin{enumerate}[topsep=0pt]

	\item Redondea a las centésimas el número 12,23563

	\item Trunca a las décimas el número 9,2934
	\item Redondea a las milésimas el número 34,19982
	\item Trunca a las centésimas el número 2,4528
	\item Redondea a las unidades el número 8,49
\end{enumerate}



\Exercicio Calcula el error absoluto y el error relativo cometido:
\begin{enumerate}
	\item Al aproximar el número $\sqrt{2}$ por $1,4$.
	\item Al aproximar el número $3,823$ por $3$.
	\item Al aproximar el número $2,456$ por $2,4$.
	\item Al aproximar el número $19231$ por $19200$.
\end{enumerate} 



\section{Notación Científica}

\Exercicio Realiza las siguientes operaciones expresando previamente los números en notación científica.
\begin{enumerate}
	\item $42 000 + 0,002$
	\item $42 000 \cdot 0,002$
	\item $2456 \cdot 10^{121} + 12345 \cdot 10^{122}$
	\item $2345000 +  124$
	\item $874000 \cdot 23$ 
\end{enumerate}

\section{Radicales y operaciones con ellos}

\Exercicio Extrae factores de los siguientes denominadores:

\begin{enumerate}[topsep=0pt]

	\item $ \sqrt{3^{10} \cdot 5^7 \cdot 7 \cdot 13^2}$

	\item $ \sqrt[3]{3^{10} \cdot 5^6 \cdot 7^2 \cdot 13^3} $
	\item $ \sqrt[4]{2^{9} \cdot 3^8 \cdot 5^3 \cdot 23^{17}} $
	\item $ \sqrt[3]{5^{10} \cdot 13^2 \cdot 7 \cdot 23^{13}} $
	\item $ \sqrt[5]{2^{10} \cdot 5^6 \cdot 7^4 \cdot 31} $


\end{enumerate}



\Exercicio Introduce los siguientes factores en el radical:

\begin{enumerate}[topsep=0pt]

	\item $ 2 \cdot 3^2 \sqrt{5}$

	\item $ 2 \cdot 3^2 \sqrt[5]{5^3}$
	\item $ 3^3 \cdot 5^4 \sqrt[3]{3^2}$
	\item $ 2^4 \cdot 3 \sqrt[4]{2^3 \cdot 3^2}$
	\item $ 5^2 \cdot 7^3 \sqrt{5^2 \cdot 7^3}$

\end{enumerate}



\Exercicio Simplifica las siguientes expresiones

\begin{enumerate}[topsep=0pt]

	\item $\sqrt{2} + \frac{3}{2} \sqrt{8} - \frac{1}{4}\sqrt{18} $
	\item $\sqrt[3]{16} + 2 \sqrt[3]{2} - \sqrt[3]{54} - \frac{21}{5} + \sqrt[3]{250}$
	\item $5 \sqrt{125} + 6\sqrt{45} - 7\sqrt{20} + \frac{3}{2} + \sqrt{80} $
	\item $\sqrt[4]{144 a^2} - 2 \sqrt{\frac{27}{16} a} + \sqrt{3a} $

	\item $\frac{a^4 \sqrt[3]{a^2}(\sqrt{a})^3}{\sqrt{\sqrt[3]{a^5}}} $

\end{enumerate}



\Exercicio Racionaliza los siguientes denominadores:

\begin{enumerate}[topsep=0pt]

	\item $ \frac{5}{2\sqrt{5}} $

	\item $ \frac{4}{\sqrt[3]{2}} $

	\item $ \frac{10}{2\sqrt[4]{5^3}} $
	\item $ \frac{6}{\sqrt[5]{3^2}} $
	\item $ \frac{5}{2\sqrt{5} + 1} $

	\item $ \frac{\sqrt{6}}{2\sqrt{3} - 3\sqrt{2}} $

	\item $ \frac{2\sqrt{3}-\sqrt{2}}{\sqrt{18}}$
	\item $ \frac{1}{2(\sqrt{3}-\sqrt{5})}$
	\item $ \frac{3}{\sqrt{5}-2}$
	\item $ \frac{3}{\sqrt{3}-\sqrt{2}}$
\end{enumerate}


\section{Logaritmos y sus propiedades}

\Exercicio Aplicando la definición, halla el valor de los logaritmos:

\begin{enumerate}[topsep=0pt]

	\item $log_3~27 $
	\item $log~1000 $
	\item $log_3~\sqrt{27}$
	\item $log_5~\sqrt[3]{25} $

	\item $log_7~\frac{1}{49} $

	\item $log_9~\sqrt[3]{3} $

	\item $log_3~0,\widehat{3} $

\end{enumerate}



\Exercicio Factoriza y aplica las propiedades de los logaritmos para expresar en función de suma ou resta de logaritmos de números primos:

\begin{enumerate}[topsep=0pt]

	\item $log~2^4 \cdot 3 - log~2^3 \cdot 3^4 $

	\item $log~24 + log~\frac{16}{9} - log~144 $

	\item $log~686 + log~56 - log~\frac{16}{49} $

	\item $log~100 - log~\frac{8}{25} + log~250 $
	
	\item $log~98 + 3 log~56 - 3 log~\frac{49}{256}$
	
	\item $log~200 - 7 log 8 + log \sqrt{\frac{256}{125}}$
	
	\item $2 log 18 - 2 log~\sqrt[3]{256} + 4 log~\frac{27}{16^3}$

\end{enumerate}



\Exercicio Expresa el valor de E en cada caso sin que aparezcan logaritmos:

\begin{enumerate}[topsep=0pt]

	\item $log~E =  log~x + 2 log~y - log~z$

	\item $log~E = 3(log~x -1) - 2(1-log~y)$

	\item $log~E = 2 log~\sqrt{x} - log~x - log~y + 3 log~\sqrt[3]{y}$

	\item $log~E = log~x^3 + 3 log~y - log~x^4$
	
	\item $log~E = 2 - 2 log~x + log~y - 5 log~2$
	
	\item $log~E = 3 log 2 - 4 log~x + 3 log~y$
	
	\item $log~E = 3 (2log~x - 3 log~y +\frac{log~z}{2})-\frac{2 log~t}{3}$
	
	\item $log~E = \frac{\frac{log~a}{3} - 2 log~b}{5} - \frac{3 log~c}{2}$

\end{enumerate}