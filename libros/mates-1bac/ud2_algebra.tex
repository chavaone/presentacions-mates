\chapter{Álgebra}
\setcounter{exercicio}{0}

\section{Polinomios}

\Exercicio Realiza las siguientes operaciones con polinomios:
\begin{enumerate}[topsep=0pt]
	\item $(2x^2-2x-13)(3x^2-2x)-3x$
	\item $(2x^2-3x+2)(-3x^2+x+1) + (6x-10)x^3$
	\item $(x^3+2x^2+3x+4)(5x-2)$
	\item $(x^2+3x -2)(x-2) - 2x - (4x^2-4)(x^3-2x+4)$
	\item $(x^3-2x+5)(x-4) - 3x^2 + 4 - x^2(x+4)$
\end{enumerate}


\Exercicio Realiza las siguientes operaciones calculando previamente las potencias:
\begin{enumerate}[topsep=0pt]
	\item $(3x-4)^3 + (2x - 4)^2 + (2x - 3)(2x+3)$
	\item $(y^2+3)^4 + (2y +3)^2$
	\item $(x^2y -3x)^3 - (2x^2y- 4x)^2$
	\item $(x^2-3)(x^2+3) + (2x^2+3)^4$
	\item $(y^3-2x)^3 + (2x^2+y)^2$
	\item $(x^2+4)(x^2-4)(x-2) + 2x(x-3)^3$
	\item $2(x-2)^2 - 3(3x+2)^3 - 2(3x-2)(3x+2)$
\end{enumerate}


\Exercicio Calcula el cociente y el resto de la división de los siguientes polinomios usando el método adecuado:
\begin{enumerate}[topsep=0pt]
	\item $(6x^4+7x^3-5x^2-6x-6):(3x^2+2x+1)$
	\item $(3x^3+10x^2-6x+5):(x+4)$
\end{enumerate}
\paragraph{División por caja}
\begin{enumerate}[topsep=0pt]
	\item $(x^5+3x^3-3x^2+6x-18):(x^2+2x+3)$
	\item $(x^5 + 1) : (x^3 + x^2 + 2x - 5)$
	\item $(3x^6 - 5x^5 + 4x^4 - 2x^3 + 3x^2 - x + 2) : (x^2 - x + 2)$
	\item $(6x^6 + 5x^5 - 3x^3 + 2x - 1) : (x^2 - 1)$
	\item $(12x^6+8x^5-3x^4-9x^3+11x^2+35x-30+6x+8):(3x^2+2x-3)$
\end{enumerate}
\paragraph{División por Ruffini}
\begin{enumerate}[topsep=0pt]
	\item $(3x^3 - 2x + 5) : (x-5)$
	\item $(2x^6+4x^5-3x^4+8x^3+4x-9) : (x+3)$
	\item $(x^5-3x^4-10x^2-22x-3) : (x -4)$
	\item $(5x^5-3x^4-9x^3+3x^2-16x-10) : (x - 2) $
	\item $(x^5+2x^3-2x-1):(2x+4)$
	\item $(2x^4+2x^3-2x^2-2):(-x+3)$
	\item $(-2x^4-x^3-2x^2+x+1):(2x-3)$
\end{enumerate}


\Exercicio Factoriza los siguientes polinomios:
\begin{enumerate}[topsep=0pt]
	\item $x^3-2x^2-5x+6$
	\item $x^3+x^2-5x+3$
	\item $x^3 + x^2$
	\item $2x^3 + 3x^2 - 2x$
	\item $x^4 + 4x^3 - 5x^2$
	\item $x^4 - 25x^2$
	\item $x^4 - 4x^3 - 12x^2$
	\item $7x^3 + 5x^2 - 2x$
\end{enumerate}

\section{Fracciones algebraicas}

\Exercicio Realiza las siguientes operaciones con fracciones algebraicas. Simplifica el resultado:
\begin{enumerate}[topsep=0pt]
	\item $\frac{1}{x-3} + \frac{3x-10}{x^2-6x+8} - \frac{2x-7}{x-4}$
	\item $\frac{x^2-1}{x+3} \cdot \frac{x^2-4}{x-1} \cdot \frac{x^2-9}{x+2}$
	\item $ \frac{1}{x^2 - 3x - 4} - \frac{2}{x-4} + \frac{5}{x+1}$
	\item $\frac{x}{2x^2 +3x - 5} - \frac{1}{x-1} - \frac{x}{2x + 5}$
	\item $\frac{x+3}{x^2-5x+4} + \frac{2x}{x-4} + \frac{1}{x-1}$
	\item $ \frac{9x}{3x-3} \cdot \frac{x^2-1}{3x^2}$
	\item $\frac{x^2 - 25}{x^2 + 25} \cdot \frac{x+5}{x-5}$
	\item $\frac{x^2-1}{x^2 - 4x +4} : \frac{x^2 + 2x + 1}{x^2 - 4}$
	\item $\frac{2x-1}{x^2 + 2x} : \frac{4x}{x^3 + 2x^2}$
\end{enumerate}

\section{Ecuaciones}

\Exercicio Resuelve las siguientes ecuaciones polinómicas:
\begin{enumerate}[topsep=0pt]
	\item $\frac{2x}{3} - \frac{x-2}{12} + \frac{x+3}{2} = 2x - \frac{1}{6}$
	\item $\frac{x+10}{2} + \frac{2(x-2)}{5} = \frac{5x-15}{3}$
	\item $4x^2-7x -2 = 0$
	\item $x(2x-1) -3x(x+1) = 0$
	\item $x^4-x^3-5x^2-x= 6$
	\item $x^4- 125x^2 + 484 = 0$
\end{enumerate}

\paragraph{Primer Grado}
\begin{enumerate}[topsep=0pt]
	\item $\frac{x}{2} + 3(x + 2) = \frac{x + 1}{3} - 2$
	\item $\frac{3(2x-3)}{2} - \frac{2(x-2)}{5} - 1 = -\frac{2}{5} - \frac{x}{2}$
	\item $\frac{5}{8} - \frac{7(x-2)}{10} = \frac{7}{4} - \frac{3(1-x)}{20}$
	\item $\frac{2(x+1)}{3} - \frac{3(2x-1)}{4} = 2 - \frac{5(x-2)}{6}$
\end{enumerate}

\paragraph{Segundo Grado}
\begin{enumerate}[topsep=0pt]
	\item $x^2 + 2x = 3x$
	\item $2x^2 + 3x + 9 = x^2 - 3x$
	\item $4x^2 + 5x + 8 = 2(x^2 + 4)$
	\item $3x^2 + 3 +2x = 2 (x + 1)$
	\item $6x^2 - 7x - 3 = 0$
\end{enumerate}

\paragraph{Cuadráticas}
\begin{enumerate}[topsep=0pt]
	\item $x^4 - 17x^2 + 16 = 0$
	\item $81x^4 - 45x + 4 = 0$
	\item $x^4 - 34x^2 + 225 = 0$
	\item $x^4 + x^2 - 6 = 0$
\end{enumerate}


\paragraph{Grado mayor que 2}
\begin{enumerate}[topsep=0pt]
	\item $x^4+x^3+8 = 6x^2 + 4x$
	\item $2x^4-6x^3+24x + 7 = 3 + 2 (4x^2 + 2)$
	\item $3x^5 - x^4 -2x^3 + x^2 - x +2 = x^5-4x^4 +x^3 + 3x^2- x +2$
	\item $x^5-4x^4-x^3+17x^2+12x = 0$
\end{enumerate}

\Exercicio Resuelve las siguientes ecuaciones racionales:

\begin{enumerate}[topsep=0pt]
	\item $ \frac{2x+3}{x^2-x} = \frac{x}{x^2 + x -2} $
	\item $ \frac{2x+1}{2x^2 +x - 1} = \frac{4x-1}{4x^2-1} $
	\item $ \frac{x+2}{x^3-x} = \frac{x+1}{x^2-2x^2+x} $
	\item $ \frac{3}{x^3+x^2-6x} = \frac{1-5x}{x^3-5x^2+6x} $
	\item $ \frac{2x-3}{x^2-x} - \frac{5-2x}{x^2+x} = \frac{3x-5}{x^2-1} $
	\item $ \frac{2}{3x^2-6x}-\frac{5}{6x^2-24} = \frac{1}{9x^2+18x}$
	\item $ \frac{x+2}{x+1} - \frac{x+1}{x+2} = \frac{9}{20}$
	\item $ \frac{x+1}{x(x-1)} + \frac{x+4}{(x+1)(x-1)} = \frac{x+1}{x} - \frac{x+2}{x(x+1)}$
	\item $ \frac{x+1}{x^2-x-6} - \frac{x-1}{x^2+4x+4} = \frac{x-1}{x^2-3x}$
\end{enumerate}


\Exercicio Resuelve las siguientes ecuaciones con radicales:
\begin{enumerate}[topsep=0pt]
	\item $ \sqrt{2x+3} + x = 6 $
	\item $ x - \sqrt{2x-1} = 2 $
	\item $ \sqrt{x+2} - \sqrt{x-1} = 1 $
	\item $ \sqrt{3x + 9} - \sqrt{2x+1} = 2 $
	\item $ 2x + \sqrt{x+2} = -1$
	\item $ \sqrt{x+5} - \sqrt{2x+3} = 1$
	\item $ \sqrt{2x-2} + \sqrt{x+3} = 2$
	\item $ \sqrt{2x+2} -x = -\sqrt{x+2}$
	\item $ \frac{\sqrt{10+2x}}{2} - 2x = - \frac{1}{2}(1+5x)$
\end{enumerate}


\Exercicio Resuelve las siguientes ecuaciones logarítmicas:

\begin{enumerate}[topsep=0pt]
	\item $ log(3+x) = log (6) -log(4-x) $
	\item $ 2 log(2x+1) - log(x+1) = log(x-7)$
	\item $ log(3x+3) - log(2x-1) = log(x+1) - log(2x+5)$
	\item $ log(x+1) = 1 - log(x-8)$
	\item $ 2log_2(x+1) = 3 + log_2(x-1)$
	\item $ 2log(2x-2)-log(x-1) = 1$
	\item $ log(65-x^3)= 3log(5-x)$
	\item $ log_3(x-1) - log_3 (x+2) = 1 - log_3(x+6)$
	\item $ log(\sqrt{7x+51}) -1 = log(9) - log(\sqrt{2x+67})$
\end{enumerate}

\Exercicio Resuelve las siguientes ecuaciones exponenciales:

\begin{enumerate}[topsep=0pt]
	\item $ 2^{8-2x} = \frac{1}{4} $
	\item $ (\sqrt{2})^{x^2+1} = \sqrt[8]{32}$
	\item $ 2^{x+2} - 2^{x+1} - 2^x + 2^{x-1} = 12$
	\item $ 3^{x+3} - 3^{x+1} - 3^{x-1} = 71 $
	\item $ 2^{x+1} = 5 $
	\item $ 5^{\frac{1}{x}} = \sqrt{2} $
	\item $ 7^{x^2-2} = \frac{1}{9}$
	\item $ 3^{3x+1} = e$
	\item $ 3^{2x-1} = \sqrt[3]{9}$
	\item $ 5^{\frac{x^2}{2}-1} = \frac{1}{\sqrt{5}}$
	\item $ 9^x - 10 \cdot 3^x + 9 = 0$
	\item $ 2^{2x} - 18 \cdot 2^x + 32$
	\item $ 81^x = \sqrt[3]{9^2}$
	\item $ 5^{x-1} + 2 \cdot 5^{x+2} = 3$
	\item $ 3^{2x+5} = 11$
\end{enumerate}


\Exercicio Resuelve los siguientes sistemas de ecuaciones por el método que creas conveniente:

\begin{enumerate}[topsep=0pt]
	\item $\begin{cases}
			x+5y = -1 \\
			3x-2y = 14
			\end{cases}$
	\item $\begin{cases}
			2x+3y = 21 \\
			-x + y = 2
			\end{cases}$
	\item $\begin{cases}
			2(x+3) - 4(x+2y-1) = 21 \\
			5(3x-y) + 3(2x+4y-2) = -6
		   \end{cases}$
	\item $\begin{cases}
			\frac{9(x-2)}{10} - \frac{4(2y + 3)}{15} = \frac{-7}{30} \\
			-\frac{5(x+4)}{12} + \frac{7(y-1)}{18} = -\frac{1}{36}
		   \end{cases} $
	\item $\begin{cases}
			\frac{3(x+1)}{2} - \frac{7(y-2)}{10} = \frac{4}{5} \\
			\frac{3(x2-1)}{8} + \frac{5(y-2)}{6} = \frac{11}{24}
		   \end{cases}$
\end{enumerate}


\Exercicio Resuelve y clasifica, según el número de soluciones, los siguientes sistemas de ecuaciones lineales empleando el método de Gauss-Jordan:

\begin{enumerate}[topsep=0pt]
	\item $ \begin{cases}
		5x+2y-z=4 \\
		2x-y+3z = -1 \\
		x+4y-7z = 6
	\end{cases} $
	
	\item $ \begin{cases}
		-x + y +2z = -2 \\
		3x-y+3z = -1 \\
		3x+y+12z = 3
	\end{cases} $
	
	\item $ \begin{cases}
		2x - y + 5z = 5 \\
		-2x -y + 4z = -3 \\
		6x -y + 6z = 13
	\end{cases} $
	
	\item $ \begin{cases}
		3x+3y-2z = 7 \\
		2x-y+5z = -1 \\
		-x+5y-12z = 3
	\end{cases} $
	
	\item $ \begin{cases}
			x-2y + 4z = 3 \\
			2x+y-3z = -11 \\
			-3x+2y+z = 13
			\end{cases} $
	
	\item $ \begin{cases}
		x+2y-3z = 5 \\
		-2x-y+2z = -3 \\
		3x+3y-5z=8
	\end{cases} $

	
	\item $ \begin{cases}
		x+y+z= 2 \\
		-2z-4y+z = 5 \\
		3x-y = -3
	\end{cases} $
	
	\item $ \begin{cases}
		3x-y+2z=4\\
		-5x-y+z=1\\
		x-3y+5z=3
	\end{cases} $

\end{enumerate}

\Exercicio Resuelve los siguientes sistemas de ecuaciones no lineales:

\begin{enumerate}[topsep=0pt]
	\item $ \begin{cases}
		x^2 - 3y^2 = 4 \\
		xy = -8
	\end{cases}  $
	\item $ \begin{cases}
		\sqrt{x+1} + 2\sqrt{y} = 4 \\
		x+3y=6
	\end{cases}  $
	\item $ \begin{cases}
		\frac{x+y}{x-y} -\frac{x-y}{x+y} = \frac{8}{3} \\
		xy = 2
	\end{cases}  $
	\item $ \begin{cases}
		\sqrt{x} - \sqrt{y} = -1 \\
		2\sqrt{x} + 3\sqrt{y} = 13
	\end{cases}  $

	\item $ \begin{cases}
		x \\
		y
	\end{cases}  $
\end{enumerate}


