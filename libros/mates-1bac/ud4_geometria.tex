\chapter{Geometría Analítica}
\setcounter{exercicio}{0}


\Exercicio Dado los puntos del plano A(-2,3); B(1,7); C(6,7) y D(3,3). Representa los siguientes vectores, calcula sus coordenadas y su módulo:

\begin{enumerate}[topsep=0pt]
	\item \textbf{[C]} $\overrightarrow{AB}$
	\item \textbf{[C]} $\overrightarrow{BA}$
	\item \textbf{[C]} $\overrightarrow{AC}$
	\item \textbf{[C]} $\overrightarrow{DC}$
	\item \textbf{[C]} $\overrightarrow{DA}$
\end{enumerate}


\Exercicio ¿Cuáles de los vectores calculados en el ejercicio 1 son equipolentes? Indica y representa un par de puntos que definan un vector equipolente a $\vec{AC}$

\Exercicio Representa y calcula de forma analítica la suma de los siguientes vectores:

\begin{enumerate}[topsep=0pt]
	\item \textbf{[C]} $\overrightarrow{u}(3,2) + \overrightarrow{u}(-1,0)$
	\item \textbf{[C]} $\overrightarrow{u}(-2,6) + \overrightarrow{u}(4,3)$
\end{enumerate}


\Exercicio Representa y calcula de forma analítica los siguientes productos escalares de vectores:
\begin{enumerate}[topsep=0pt]
	\item \textbf{[C]} $ 4 \cdot \overrightarrow{u}(1,1)$
	\item \textbf{[C]} $ 0,5 \cdot \overrightarrow{u}(-2,6)$
	\item \textbf{[C]} $ -2 \cdot \overrightarrow{u}(1,3)$
\end{enumerate}


\Exercicio Dados los puntos A(2,4), B(5,8) y C(6,4), calcula las coordenadas de los siguientes vectores:

\begin{enumerate}[topsep=0pt]
	\item \textbf{[C]} $ \vec{u} = 2 \cdot \vec{AB} + 3 \cdot \vec{AC} $
	\item \textbf{[C]} $ \vec{v} = 4 \cdot \vec{BA} - 3 \cdot \vec{CA} $
	\item \textbf{[C]} $ \vec{z} = 3 \cdot \vec{BC} - 5 \cdot \vec{BA} $
\end{enumerate}


\Exercicio Calcula los valores de $x$ e $y$ para que se verifiquen las siguientes igualdades:

\begin{enumerate}[topsep=0pt]
	\item \textbf{[C]} $ (15, 4) = 2(x, 4) + 3(2, y) $
	\item \textbf{[C]} $ (-4, y) = 3(7, 5) + 3(x, -y) $
\end{enumerate}


\Exercicio Indica si son linealmente independientes los siguientes conjuntos de vectores.

\begin{enumerate}[topsep=0pt]
	\item \textbf{[C]} $\vec{u}(2,3)$ y $\vec{w}(0,8)$
	\item \textbf{[C]} $\vec{u}(1,4)$ y $\vec{w}(3, 12)$
	\item \textbf{[C]} $\vec{u}(3,5)$ y $\vec{w}(2,7)$
	\item \textbf{[C]} $\vec{u}(3,5)$, $\vec{v}(0,3)$ y $\vec{w}(6,16)$
	\item \textbf{[C]} $\vec{u}(2,7)$, $\vec{v}(4,14)$ y $\vec{w}(3,21)$
\end{enumerate}


\Exercicio Determina si los vectores $\vec{u}$ y $\vec{v}$ forman una base. En caso de que formen una base, indica las coordenadas del vector $\vec{a}$ para dicha base:

\begin{enumerate}[topsep=0pt]
	\item \textbf{[C]} $\mathfrak{B} = \{\vec{u}(1,2), \vec{v}(2,4)\}$ y $\vec{a} = (5,7)$
	\item \textbf{[C]} $\mathfrak{B} = \{\vec{u}(0,3), \vec{v}(1,2)\}$ y $\vec{a} = (1,2)$
	\item \textbf{[C]} $\mathfrak{B} = \{\vec{u}(3,1), \vec{v}(2,7)\}$ y $\vec{a} = (8,9)$
	\item \textbf{[C]} $\mathfrak{B} = \{\vec{u}(3,5), \vec{v}(1,5)\}$ y $\vec{a} = (2,5)$
\end{enumerate}


\Exercicio Determina las coordenadas de los siguientes vectores en la Base Canónica:

\begin{enumerate}[topsep=0pt]
	\item \textbf{[C]} $ (2,5)_\mathfrak{\mathfrak{B}} $ con $\mathfrak{B} = \{\vec{u}(2,3), \vec{j}(1,4) \} $
	\item \textbf{[C]} $ (1,-4)_\mathfrak{B} $ con $\mathfrak{B} = \{\vec{u}(1,3), \vec{j}(3,2) \} $
\end{enumerate}


\Exercicio Determina si los siguientes vectores forman una base y si esta es ortogonal y/o ortonormal:

\begin{enumerate}[topsep=0pt]
	\item \textbf{[C]} $ \vec{i}(4,5)$ y $\vec{j}(-5,4) $
	\item \textbf{[C]} $ \vec{i}(1,0)$ y $\vec{j}(-1,0) $
	\item \textbf{[C]} $ \vec{i}(-1,0)$ y $\vec{j}(0,-1) $
\end{enumerate}


\Exercicio Calcula la distancia entre los siguientes pares de puntos:

\begin{enumerate}[topsep=0pt]
	\item \textbf{[C]} A(0,3) y B(7,8)
	\item \textbf{[C]} A(2,7) y B(1,5)
\end{enumerate}


\Exercicio Calcula el punto medio del segmento determinado por los siguientes pares de puntos:

\begin{enumerate}[topsep=0pt]
	\item \textbf{[C]} A(0,3) y B(7,8)
	\item \textbf{[C]} A(2,7) y B(1,5)
\end{enumerate}


\Exercicio Escribe la ecuacion de la recta en todas sus formas e indicando los nombres en cada uno de los siguientes casos:

\begin{enumerate}[topsep=0pt]
	\item \textbf{[C]} Vector director $\vec{v}(2,3)$ y pasa por A(1,5).
	\item \textbf{[C]} Pasa por $A(2,5)$ y por $B(3,7)$.
	\item \textbf{[C]} Tiene por vector normal $\vec{n}(2,-4)$ y pasa por $A(2,7)$.
	\item \textbf{[C]} Tiene por pendiente $m=3$ y pasa por $A(1,9)$.
	\item \textbf{[C]} Forma un ángulo $\alpha=45º$ con el eje de abscisas (el de la x's) y pasa por $A(2,3)$
\end{enumerate}


\Exercicio Dada la recta $r \equiv 2x + 3y -7 = 0$ calcula:

\begin{enumerate}[topsep=0pt]
	\item \textbf{[C]} El vector director y la pendiente.
	\item \textbf{[C]} La ecuación general de la recta que pasa paralela a r y que pasa por $P(3,1)$.
	\item \textbf{[C]} El vector normal y la pendiente de la recta perpendicular a r.
	\item \textbf{[C]} La ecuación general de la recta perpendicular arque pasa por el punto $Q(1,-5)$
\end{enumerate}


\Exercicio Estudia la posición relativa de las siguientes rectas:

\begin{enumerate}[topsep=0pt]
	\item \textbf{[C]} $r \equiv 3x-5y+ 4 = 0 $ y $s \equiv 3x+ 5y-3 = 0 $
	\item \textbf{[C]} $r \equiv x-3y+ 1 = 0$ y $s \equiv -2x+ 6y-2 = 0$
	\item \textbf{[C]} $r \equiv y = 2x+ 3$ y $s \equiv x= \frac{y}{2}$
\end{enumerate}


\Exercicio Calcula el producto escalar entre los siguientes pares de vectores:

\begin{enumerate}[topsep=0pt]
\item \textbf{[C]} $ \vec{u}(2,3)$ y $\vec{v}(1,7) $
	\item \textbf{[C]} $ \vec{u}(0,3)$ y $\vec{v}(1,8) $
	\item \textbf{[C]} $ \vec{u}(4,1)$ y $\vec{v}(2,7) $
\end{enumerate}


\Exercicio Comprueba si los siguientes pares de vectores son perpendiculares de forma analítica:

\begin{enumerate}[topsep=0pt]
\item \textbf{[C]} $ \vec{u}(0,3)$ y $\vec{v}(1,0) $
	\item \textbf{[C]} $ \vec{u}(3,1)$ y $\vec{v}(-2,6) $
	\item \textbf{[C]} $ \vec{u}(1,4)$ y $\vec{v}(4,1) $
\end{enumerate}


\Exercicio Calcula el módulo de las siguientes proyecciones:

\begin{enumerate}[topsep=0pt]
\item \textbf{[C]} Proyección del vector $\vec{v}(3,5)$ sobre el vector $\vec{u}(1,10)$
\end{enumerate}


\Exercicio Calcula el ángulo que hay entre los siguientes pares de rectas:

\begin{enumerate}[topsep=0pt]
\item \textbf{[C]} $ r \equiv y = 0 $ y $ s \equiv x = 0 $
	\item \textbf{[C]} $ r \equiv y = 3x + 5  $ y $ s \equiv 2x + 3y +5 = 0 $
	\item \textbf{[C]} $ r \equiv y = 2x - 3  $ y $ s \equiv x = 0 $
\end{enumerate}


\Exercicio Dados los vectores $ \vec{u}(3,-4)$ y $ \vec{v} (5,6) $ calcula:

\begin{enumerate}[topsep=0pt]
\item \textbf{[C]} Su producto escalar.
	\item \textbf{[C]} El módulo de los dos vectores.
	\item \textbf{[C]} El ángulo que forman.
	\item \textbf{[C]} Un vector de la misma dirección y sentido que $\vec{u}$ y que sea unitario.
	\item \textbf{[C]} Un vector que tenga la misma dirección pero sentido contrario a $\vec{u}$ y que tenga módulo 3.
	\item \textbf{[C]} Un vector ortogonal a $\vec{u}$.
\end{enumerate}


\Exercicio Calcula la distancia entre las rectas y puntos dados:

\begin{enumerate}[topsep=0pt]
\item \textbf{[C]} A(2,3) y $ r \equiv x = 0 $
	\item \textbf{[C]} A(5,6) y $ r \equiv y = 2x + 3 $
\end{enumerate}


\Exercicio Indica la distancia entre las siguientes rectas:

\begin{enumerate}[topsep=0pt]
\item \textbf{[C]} $ r \equiv 2x + 3y +4 = 0 $ y $ s \equiv 4x + 6y - 4 = 0 $
	\item \textbf{[C]} $ r \equiv y = 2x + 3 $ y $ s \equiv y = 3x - 5 $
\end{enumerate}


\Exercicio Calcula $a$ y $b$ para que las rectas  $r \equiv 3x-ay -2b = 0$ y $s \equiv 3x-6y +4 = 0$ cumplan las siguientes propiedades:

\begin{enumerate}[topsep=0pt]
\item \textbf{[C]} Sean paralelas. Calcula su distancia.
	\item \textbf{[C]} Sean perpendiculares.
	\item \textbf{[C]} Formen un ángulo de 45º.
\end{enumerate}


\Exercicio Para cada apartado, calcula el simétrico de cada punto con respecto a la recta dada:

\begin{enumerate}[topsep=0pt]
\item \textbf{[C]} A(2,3) y $ r \equiv x = 0 $
	\item \textbf{[C]} A(5,6) y $ r \equiv y = 2x + 3 $
\end{enumerate}


\Exercicio Calcula la mediatriz de los segmentos delimitados por A y B en cada caso:

\begin{enumerate}[topsep=0pt]
\item \textbf{[C]} A(4,0) y B(10,0)
	\item \textbf{[C]} A(2,3) y B(3,7)
\end{enumerate}


\Exercicio Calcula las bisectrices de los siguientes pares de rectas:

\begin{enumerate}[topsep=0pt]
\item \textbf{[C]} $ r \equiv x=0 $ y $ s \equiv y = 0 $
	\item \textbf{[C]} $ r \equiv y = 3x - 5 $ y $ s \equiv y = 5x + 9 $
\end{enumerate}


\Exercicio Dado el triángulo formado por los puntos A(0,0), B(8,0) y C(8,10), calcula:

\begin{enumerate}[topsep=0pt]
\item \textbf{[C]} Clasifica el triángulo en función de sus ángulos y lados.
	\item \textbf{[C]} Las ecuaciones de los lados
	\item \textbf{[C]} El circuncentro
	\item \textbf{[C]} El radio de la circunferencia circunscrita
	\item \textbf{[C]} La ecuación general de la mediana que parte de B
	\item \textbf{[C]} El baricentro
	\item \textbf{[C]} La longitud de la altura que pasa por A
	\item \textbf{[C]} Su área
\end{enumerate}


\Exercicio Dado el cuadrilátero formado por los puntos $A(1,1)$, $B(5,2)$, $C(3,3)$ y $D(1, 5/2)$, calcula:

\begin{enumerate}[topsep=0pt]
\item \textbf{[C]} Demuestra que es un trapecio.
	\item \textbf{[C]} Indica que tipo de trapecio es.
	\item \textbf{[C]} Calcula el punto donde se cortan las diagonales.
\end{enumerate}


