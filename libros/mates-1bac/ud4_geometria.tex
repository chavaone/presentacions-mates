\chapter{Geometría Analítica}
\setcounter{exercicio}{0}


\Exercicio Dado los puntos del plano A(-2,3); B(1,7); C(6,7) y D(3,3). Representa los siguientes vectores, calcula sus coordenadas y su módulo:

\begin{enumerate}[topsep=0pt]
	\item \textbf{[C]} $\overrightarrow{AB}$
	\item \textbf{[C]} $\overrightarrow{BA}$
	\item \textbf{[C]} $\overrightarrow{AC}$
	\item \textbf{[C]} $\overrightarrow{DC}$
	\item \textbf{[C]} $\overrightarrow{DA}$
	\item $\overrightarrow{DB}$
\end{enumerate}


\Exercicio ¿Cuáles de los vectores calculados en el ejercicio 1 son equipolentes? Indica y representa un par de puntos que definan un vector equipolente a:
\begin{enumerate}[topsep=0pt]
	\item \textbf{[C]}  $\overrightarrow{AC}$
	\item $\overrightarrow{AB}$
	\item $\overrightarrow{DB}$
\end{enumerate}


\Exercicio Representa y calcula de forma analítica la suma de los siguientes vectores:

\begin{enumerate}[topsep=0pt]
	\item \textbf{[C]} $\overrightarrow{u}(3,2) + \overrightarrow{u}(-1,0)$
	\item \textbf{[C]} $\overrightarrow{u}(-2,6) + \overrightarrow{u}(4,3)$
	
	\item $\overrightarrow{u}(4,5) + \overrightarrow{u}(-2,0)$
	\item $\overrightarrow{u}(-3,6) + \overrightarrow{u}(2,1)$
	\item $\overrightarrow{u}(1,4) + \overrightarrow{u}(3,3)$
	\item $\overrightarrow{u}(-1,-3) + \overrightarrow{u}(4,3)$
\end{enumerate}


\Exercicio Representa y calcula de forma analítica los siguientes productos escalares de vectores:
\begin{enumerate}[topsep=0pt]
	\item \textbf{[C]} $ 4 \cdot \overrightarrow{u}(1,1)$
	\item \textbf{[C]} $ 0,5 \cdot \overrightarrow{u}(-2,6)$
	\item \textbf{[C]} $ -2 \cdot \overrightarrow{u}(1,3)$
	
	\item $ 3 \cdot \overrightarrow{u}(1,2)$
	\item $ \frac{1}{3} \cdot \overrightarrow{u}(6,9)$
	\item $ -3 \cdot \overrightarrow{u}(2,0)$
	\item $ 2 \cdot \overrightarrow{u}(1,4)$
\end{enumerate}


\Exercicio Dados los puntos A(2,4), B(5,8), C(6,4) y D(2,9) calcula las coordenadas de los siguientes vectores:

\begin{enumerate}[topsep=0pt]
	\item \textbf{[C]} $ \overrightarrow{u} = 2 \cdot \overrightarrow{AB} + 3 \cdot \overrightarrow{AC} $
	\item \textbf{[C]} $ \overrightarrow{v} = 4 \cdot \overrightarrow{BA} - 3 \cdot \overrightarrow{CA} $
	\item \textbf{[C]} $ \overrightarrow{z} = 3 \cdot \overrightarrow{BC} - 5 \cdot \overrightarrow{BA} $
	
	\item $ \overrightarrow{t} = 4 \cdot \overrightarrow{AB} - 2 \cdot \overrightarrow{AD} $
	\item $ \overrightarrow{w} = 3 \cdot \overrightarrow{BC} - \cdot \overrightarrow{CD} $
	\item $ \overrightarrow{a} = 2 \cdot \overrightarrow{AD} - 3 \cdot \overrightarrow{BC} $
	\item $ \overrightarrow{c} = 6 \cdot \overrightarrow{DC} - 2 \cdot \overrightarrow{CA} $	
\end{enumerate}


\Exercicio Calcula los valores de $x$ e $y$ para que se verifiquen las siguientes igualdades:

\begin{enumerate}[topsep=0pt]
	\item \textbf{[C]} $ (15, 4) = 2(x, 4) + 3(2, y) $
	\item \textbf{[C]} $ (-4, y) = 3(7, 5) + 3(x, -y) $
	
	
	\item $ (13, 21) = 4 \cdot (x, 7) - (10, y) $
	\item $ (x, 10) = -2 \cdot (3, 5) - 3(1, 3y) $
	\item $ (4, y) = 2 \cdot (x, 6) + 3(5, -2y) $
	\item $ (x, y) = 5 \cdot (2x, 8) - 3(10, -y) $
\end{enumerate}


\Exercicio Indica si son linealmente independientes los siguientes conjuntos de vectores.

\begin{enumerate}[topsep=0pt]
	\item \textbf{[C]} $\overrightarrow{u}(2,3)$ y $\overrightarrow{w}(0,8)$
	\item \textbf{[C]} $\overrightarrow{u}(1,4)$ y $\overrightarrow{w}(3, 12)$
	\item \textbf{[C]} $\overrightarrow{u}(3,5)$ y $\overrightarrow{w}(2,7)$
	\item \textbf{[C]} $\overrightarrow{u}(3,5)$, $\overrightarrow{v}(0,3)$ y $\overrightarrow{w}(6,16)$
	\item \textbf{[C]} $\overrightarrow{u}(2,7)$, $\overrightarrow{v}(4,14)$ y $\overrightarrow{w}(3,21)$
	
	\item $\overrightarrow{u}(3,7)$ y $\overrightarrow{v}(9,21)$
	\item $\overrightarrow{u}(5,1)$ y $\overrightarrow{v}(10,3)$
	\item $\overrightarrow{u}(4,10)$ y $\overrightarrow{v}(6,15)$
	\item $\overrightarrow{u}(1,7)$, $\overrightarrow{v}(3,5)$ y $\overrightarrow{w}(5,19)$
	\item $\overrightarrow{u}(4,8)$, $\overrightarrow{v}(3,6)$ y $\overrightarrow{w}(1,7)$
	\item $\overrightarrow{u}(1,5)$, $\overrightarrow{v}(5,1)$ y $\overrightarrow{w}(11,7)$
	
\end{enumerate}


\Exercicio Determina si los vectores $\overrightarrow{u}$ y $\overrightarrow{v}$ forman una base. En caso de que formen una base, indica las coordenadas del vector $\overrightarrow{a}$ para dicha base:

\begin{enumerate}[topsep=0pt]
	\item \textbf{[C]} $\mathfrak{B} = \{\overrightarrow{u}(1,2), \overrightarrow{v}(2,4)\}$ y $\overrightarrow{a} = (5,7)$
	\item \textbf{[C]} $\mathfrak{B} = \{\overrightarrow{u}(0,3), \overrightarrow{v}(1,2)\}$ y $\overrightarrow{a} = (1,2)$
	\item \textbf{[C]} $\mathfrak{B} = \{\overrightarrow{u}(3,1), \overrightarrow{v}(2,7)\}$ y $\overrightarrow{a} = (8,9)$
	\item \textbf{[C]} $\mathfrak{B} = \{\overrightarrow{u}(3,5), \overrightarrow{v}(1,5)\}$ y $\overrightarrow{a} = (2,5)$
	
	\item $\mathfrak{B} = \{\overrightarrow{u}(1,5), \overrightarrow{v}(3,1)\}$ y $\overrightarrow{a} = (5,11)$
	\item $\mathfrak{B} = \{\overrightarrow{u}(0,1), \overrightarrow{v}(0,3)\}$ y $\overrightarrow{a} = (1,2)$
	\item $\mathfrak{B} = \{\overrightarrow{u}(2,4), \overrightarrow{v}(1,3)\}$ y $\overrightarrow{a} = (2,5)$
	\item $\mathfrak{B} = \{\overrightarrow{u}(7,1), \overrightarrow{v}(2,2)\}$ y $\overrightarrow{a} = (27,21)$
\end{enumerate}


\Exercicio Determina las coordenadas de los siguientes vectores en la Base Canónica:

\begin{enumerate}[topsep=0pt]
	\item \textbf{[C]} $ (2,5)_\mathfrak{\mathfrak{B}} $ con $\mathfrak{B} = \{\overrightarrow{u}(2,3), \overrightarrow{j}(1,4) \} $
	\item \textbf{[C]} $ (1,-4)_\mathfrak{B} $ con $\mathfrak{B} = \{\overrightarrow{u}(1,3), \overrightarrow{j}(3,2) \} $
	
	\item $ (4,2)_\mathfrak{B} $ con $\mathfrak{B} = \{\overrightarrow{u}(1,2), \overrightarrow{j}(3,3) \} $
	\item $ (1,3)_\mathfrak{B} $ con $\mathfrak{B} = \{\overrightarrow{u}(3,0), \overrightarrow{j}(1,7) \} $
	\item $ (5,1)_\mathfrak{B} $ con $\mathfrak{B} = \{\overrightarrow{u}(0,1), \overrightarrow{j}(1,0) \} $
\end{enumerate}


\Exercicio Determina si los siguientes vectores forman una base y si esta es ortogonal y/o ortonormal:

\begin{enumerate}[topsep=0pt]
	\item \textbf{[C]} $ \overrightarrow{i}(4,5)$ y $\overrightarrow{j}(-5,4) $
	\item \textbf{[C]} $ \overrightarrow{i}(1,0)$ y $\overrightarrow{j}(-1,0) $
	\item \textbf{[C]} $ \overrightarrow{i}(-1,0)$ y $\overrightarrow{j}(0,-1) $
	
	\item $ \overrightarrow{i}(3,2)$ y $\overrightarrow{j}(-2,3) $
	\item $ \overrightarrow{i}(0,1)$ y $\overrightarrow{j}(1,0) $
\end{enumerate}


\Exercicio Calcula la distancia entre los siguientes pares de puntos:

\begin{enumerate}[topsep=0pt]
	\item \textbf{[C]} A(0,3) y B(7,8)
	\item \textbf{[C]} A(2,7) y B(1,5)
	
	\item A(1,5) y B(4,8)
	\item A(3,8) y B(3,25)
	
\end{enumerate}


\Exercicio Calcula el punto medio del segmento determinado por los siguientes pares de puntos:

\begin{enumerate}[topsep=0pt]
	\item \textbf{[C]} A(0,3) y B(7,8)
	\item \textbf{[C]} A(2,7) y B(1,5)
	
		
	\item A(3,7) y B(9,5)
	\item A(4,1) y B(8,13)
\end{enumerate}


\Exercicio Escribe la ecuación de la recta en todas sus formas e indicando los nombres en cada uno de los siguientes casos:

\begin{enumerate}[topsep=0pt]
	\item \textbf{[C]} Vector director $\overrightarrow{v}(2,3)$ y pasa por A(1,5).
	\item \textbf{[C]} Pasa por $A(2,5)$ y por $B(3,7)$.
	\item \textbf{[C]} Tiene por vector normal $\overrightarrow{n}(2,-4)$ y pasa por $A(2,7)$.
	\item \textbf{[C]} Tiene por pendiente $m=3$ y pasa por $A(1,9)$.
	\item \textbf{[C]} Forma un ángulo $\alpha=45º$ con el eje de abscisas (el de la x's) y pasa por $A(2,3)$
	
	\item Vector director $\overrightarrow{v}(1,5)$ y pasa por A(2,3).
	\item Pasa por $A(1,6)$ y por $B(2,8)$.
	\item Tiene por vector normal $\overrightarrow{n}(-1,2)$ y pasa por $A(4,5)$.
	\item Tiene por pendiente $m=2$ y pasa por $A(2,1)$.
	\item Forma un ángulo $\alpha=30º$ con el eje de abscisas (el de la x's) y pasa por $A(4,4)$
	\item Tiene por vector normal $\overrightarrow{n}(-4,-3)$ y pasa por $A(1,6)$.
\end{enumerate}


\Exercicio Calcula empleando los conceptos sobre rectas vistos en clase:

\begin{enumerate}[topsep=0pt]

	\item \textbf{[C]}  Dada la recta $r \equiv 2x + 3y -7 = 0$ calcula:
	\begin{enumerate}[topsep=0pt, label=\arabic*)]
		\item El vector director y la pendiente.
		\item La ecuación general de la recta que pasa paralela a r y que pasa por $P(3,1)$.
		\item El vector normal y la pendiente de la recta perpendicular a r.
		\item La ecuación general de la recta perpendicular arque pasa por el punto $Q(1,-5)$
	\end{enumerate}


	\item  Dada la recta $r \equiv x - 2y +10 = 0$ calcula:
	\begin{enumerate}[topsep=0pt, label=\arabic*)]
		\item El vector director y la pendiente.
		\item La ecuación general de la recta que pasa paralela a r y que pasa por $P(1,10)$.
		\item El vector normal y la pendiente de la recta perpendicular a r.
		\item La ecuación general de la recta perpendicular arque pasa por el punto $Q(3,-6)$
		\item El ángulo que forma la recta con el eje de las x's.
	\end{enumerate}


	\item  Dada la recta $r \equiv y = 5x -3$ calcula:
	\begin{enumerate}[topsep=0pt, label=\arabic*)]
		\item El vector director y la pendiente.
		\item La ecuación general de la recta que pasa paralela a r y que pasa por $P(3,6)$.
		\item El vector normal y la pendiente de la recta perpendicular a r.
		\item La ecuación general de la recta perpendicular arque pasa por el punto $Q(1,1)$
		\item El ángulo que forma la recta con el eje de las x's.
	\end{enumerate}
	
\end{enumerate}


\Exercicio Estudia la posición relativa de las rectas $r$ y $s$ en cada caso:

\begin{enumerate}[topsep=0pt]
	\item \textbf{[C]} $r \equiv 3x-5y+ 4 = 0 $ ; $s \equiv 3x+ 5y-3 = 0 $
	\item \textbf{[C]} $r \equiv x-3y+ 1 = 0$ ; $s \equiv -2x+ 6y-2 = 0$
	\item \textbf{[C]} $r \equiv y = 2x+ 3$ ; $s \equiv x= \frac{y}{2}$
	
	\item  $r \equiv  y = 2x + 3$ ; $s \equiv y = 2x - 5  $
	\item  $r \equiv 2x + 3y - 4 = 0$ ; $s \equiv 4x +6y = 8$
	\item  $r \equiv 2x + 5y - 8 = =$ ; $s \equiv 5x + y -3 = 0$
\end{enumerate}


\Exercicio Calcula el producto escalar entre los siguientes pares de vectores:

\begin{enumerate}[topsep=0pt]
	\item \textbf{[C]} $ \overrightarrow{u}(2,3) \cdot \overrightarrow{v}(1,7) $
	\item \textbf{[C]} $ \overrightarrow{u}(0,3) \cdot \overrightarrow{v}(1,8) $
	\item \textbf{[C]} $ \overrightarrow{u}(4,1) \cdot \overrightarrow{v}(2,7) $
	
	\item $ \overrightarrow{u}(3,7) \cdot \overrightarrow{v}(4,-2) $
	\item $ \overrightarrow{u}(0,1) \cdot \overrightarrow{v}(2,3) $
	\item $ \overrightarrow{u}(1,1) \cdot \overrightarrow{v}(-2,-1) $
\end{enumerate}


\Exercicio Comprueba si los siguientes pares de vectores son perpendiculares de forma analítica:

\begin{enumerate}[topsep=0pt]
	\item \textbf{[C]} $ \overrightarrow{u}(0,3)$ y $\overrightarrow{v}(1,0) $
	\item \textbf{[C]} $ \overrightarrow{u}(3,1)$ y $\overrightarrow{v}(-2,6) $
	\item \textbf{[C]} $ \overrightarrow{u}(1,4)$ y $\overrightarrow{v}(4,1) $

	\item $ \overrightarrow{u}(2,3)$ y $\overrightarrow{v}(6,-4) $
	\item $ \overrightarrow{u}(4,0)$ y $\overrightarrow{v}(0,1) $
	\item $ \overrightarrow{u}(3,4)$ y $\overrightarrow{v}(0,1) $
\end{enumerate}


\Exercicio Calcula el módulo de las siguientes proyecciones:

\begin{enumerate}[topsep=0pt]
	\item \textbf{[C]} Proyección del vector $\overrightarrow{v}(3,5)$ sobre el vector $\overrightarrow{u}(1,10)$
	\item Proyección del vector $\overrightarrow{v}(1,2)$ sobre el vector $\overrightarrow{u}(0,3)$
	\item Proyección del vector $\overrightarrow{v}(1,8)$ sobre el vector $\overrightarrow{u}(4,5)$
	\item Proyección del vector $\overrightarrow{v}(2,6)$ sobre el vector $\overrightarrow{u}(1,4)$
\end{enumerate}


\Exercicio Calcula el ángulo que hay entre los siguientes pares de rectas:

\begin{enumerate}[topsep=0pt]
	\item \textbf{[C]} $ r \equiv y = 0 $ ; $ s \equiv x = 0 $
	\item \textbf{[C]} $ r \equiv y = 3x + 5  $ ; $ s \equiv 2x + 3y +5 = 0 $
	\item \textbf{[C]} $ r \equiv y = 2x - 3  $ ; $ s \equiv x = 0 $
	
	\item $ r \equiv 2x + 3y - 4 = 0 $ ; $ s \equiv 3x - 2y + 10 = 0$
	\item $ r \equiv y = 2x - 6 $ ; $s \equiv 3x - y + 3 = 0 $
	\item $ r \equiv 10x +5y -3 = 0$ ; $ s \equiv y = 0 $
\end{enumerate}


\Exercicio Calcula usando las propiedades de los vectores y del producto escalar vistas en clase:

\begin{enumerate}[topsep=0pt]
	
	\item \textbf{[C]}  Dados los vectores $ \overrightarrow{u}(3,-4)$ y $ \overrightarrow{v} (5,6) $ calcula:
	\begin{enumerate}[topsep=0pt]
		\item Su producto escalar.
		\item El módulo de los dos vectores.
		\item El ángulo que forman.
		\item Las coordenadas de un vector con la misma dirección y sentido que $\overrightarrow{u}$ y que sea unitario.
		\item Las coordenadas de un vector que tenga la misma dirección pero sentido contrario a $\overrightarrow{u}$ y que tenga módulo 3.
		\item Un vector ortogonal a $\overrightarrow{u}$.
	\end{enumerate}


	\item Dados los vectores $ \overrightarrow{u}(1,5)$ y $ \overrightarrow{v} (2,6) $:
	\begin{enumerate}[topsep=0pt]
		\item El producto escalar de $\overrightarrow{u} \cdot \overrightarrow{v}$.
		\item El módulo de los dos vectores.
		\item El ángulo que forman.
		\item Las coordenadas de un vector con la misma dirección  pero con sentido contrario a $\overrightarrow{v}$ y que sea unitario.
		\item Las coordenadas de un vector que tenga la misma dirección y sentido que $\overrightarrow{u}$ y que tenga módulo 6.
		\item Un vector ortogonal a $\overrightarrow{v}$.
		\item Comprueba si $\vec{u}$ y $\vec{v}$ son ortogonales.
		\item Comprueba si $\vec{u}$ y $\vec{v}$ forman una base. En caso afirmativo calcula las coordenadas de del vector $\overrightarrow{t}(4,16)$
		\item El módulo de la proyección de $\vec{u}$ sobre $\vec{v}$.
	\end{enumerate}


	\item Dados los vectores $ \overrightarrow{u}(1,-7)$ y $ \overrightarrow{v} (14,2) $ calcula:
	\begin{enumerate}[topsep=0pt]
		\item El producto escalar de $\overrightarrow{u} \cdot \overrightarrow{v}$.
		\item El módulo de los dos vectores.
		\item El ángulo que forman.
		\item Las coordenadas de un vector con la misma dirección  pero con sentido contrario a $\overrightarrow{v}$ y que sea unitario.
		\item Las coordenadas de un vector que tenga la misma dirección y sentido que $\overrightarrow{u}$ y que tenga módulo 2.
		\item Un vector ortogonal a $\overrightarrow{v}$.
		\item Comprueba si $\vec{u}$ y $\vec{v}$ son ortogonales.
		\item Comprueba si $\vec{u}$ y $\vec{v}$ forman una base. En caso afirmativo calcula las coordenadas de del vector $\overrightarrow{t}(13,9)$
	\end{enumerate}


\end{enumerate}

\Exercicio Calcula la distancia entre las rectas y puntos dados:
\begin{enumerate}[topsep=0pt]
	\item \textbf{[C]} A(2,3) y $ r \equiv x = 0 $
	\item \textbf{[C]} A(5,6) y $ r \equiv y = 2x + 3 $
	
	\item A(1,7) y $ r \equiv 2x + 3y +2 = 0 $
	\item A(2,5) y $ r \equiv x - 3y = 0 $
	\item A(-2,1) y $ r \equiv y = 4x - 5 $
\end{enumerate}


\Exercicio Indica la distancia entre las siguientes rectas:

\begin{enumerate}[topsep=0pt]
	\item \textbf{[C]} $ r \equiv 2x + 3y +4 = 0 $ ; $ s \equiv 4x + 6y - 4 = 0 $
	\item \textbf{[C]} $ r \equiv y = 2x + 3 $ y $ ; \equiv y = 3x - 5 $
	
	\item $ r \equiv 2x - y +3 $ ; $ s \equiv y = \frac{4x + 7}{2}  $
	\item $ r \equiv 3x - 5y +2 = 0 $ ; $ s \equiv 4x + 2y = 0  $
\end{enumerate}


\Exercicio Calcula los posibles valores de los parámetros $a$ y $b$ para que se cumplan las propiedades:
\begin{enumerate}[topsep=0pt]
	\item \textbf{[C]} Dadas las rectas $r \equiv 3x-ay -2b = 0$ y $s \equiv 3x-6y +4 = 0$:
	\begin{enumerate}[topsep=0pt]
		\item Sean paralelas. Calcula su distancia.
		\item Sean perpendiculares.
		\item Formen un ángulo de 45º.
	\end{enumerate}

	\item Dadas las rectas $r \equiv 2x + ay + 3 = 0$ y $s \equiv y = x - b$:
	\begin{enumerate}[topsep=0pt]
		\item Sean paralelas. Calcula su distancia (en función del valor de b).
		\item Sean perpendiculares.
		\item Formen un ángulo de 30º.
	\end{enumerate}

	\item Dadas las rectas $r \equiv ax + y - 10 = 0$ y $s \equiv 2x + y + 2b = 0$:
	\begin{enumerate}[topsep=0pt]
		\item Sean paralelas. Calcula su distancia (en función del valor de b).
		\item Sean perpendiculares.
		\item Formen un ángulo de 60º.
	\end{enumerate}

\end{enumerate}



\Exercicio Calcula el simétrico de cada punto con respecto a la recta dada:

\begin{enumerate}[topsep=0pt]
	\item \textbf{[C]} A(2,3) y $ r \equiv x = 0 $
	\item \textbf{[C]} A(5,6) y $ r \equiv y = 2x + 3 $
	
	\item A(3,6) y $ r \equiv 2x + y -3 = 0 $
	\item A(1,2) y $ r \equiv y = 4x $
	\item A(4,5) y $ r \equiv 2x - y + 1 = 0 $
\end{enumerate}


\Exercicio Calcula la mediatriz de los segmentos delimitados por A y B en cada caso:

\begin{enumerate}[topsep=0pt]
\item \textbf{[C]} A(4,0) y B(10,0)
	\item \textbf{[C]} A(2,3) y B(3,7)
\end{enumerate}


\Exercicio Calcula las bisectrices de los siguientes pares de rectas:

\begin{enumerate}[topsep=0pt]
\item \textbf{[C]} $ r \equiv x=0 $ y $ s \equiv y = 0 $
	\item \textbf{[C]} $ r \equiv y = 3x - 5 $ y $ s \equiv y = 5x + 9 $
\end{enumerate}


\Exercicio Dado el triángulo formado por los puntos A(0,0), B(8,0) y C(8,10), calcula:

\begin{enumerate}[topsep=0pt]
\item \textbf{[C]} Clasifica el triángulo en función de sus ángulos y lados.
	\item \textbf{[C]} Las ecuaciones de los lados
	\item \textbf{[C]} El circuncentro
	\item \textbf{[C]} El radio de la circunferencia circunscrita
	\item \textbf{[C]} La ecuación general de la mediana que parte de B
	\item \textbf{[C]} El baricentro
	\item \textbf{[C]} La longitud de la altura que pasa por A
	\item \textbf{[C]} Su área
\end{enumerate}


\Exercicio Dado el cuadrilátero formado por los puntos $A(1,1)$, $B(5,2)$, $C(3,3)$ y $D(1, 5/2)$, calcula:

\begin{enumerate}[topsep=0pt]
\item \textbf{[C]} Demuestra que es un trapecio.
	\item \textbf{[C]} Indica que tipo de trapecio es.
	\item \textbf{[C]} Calcula el punto donde se cortan las diagonales.
\end{enumerate}


