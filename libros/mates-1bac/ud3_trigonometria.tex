\chapter{Trigonometría}
\setcounter{exercicio}{0}

\section{Ángulos}

\Exercicio Expresa en radianes las siguientes medidas angulares:

\begin{enumerate}[topsep=0pt]
	\item 30\textdegree
	\item 60\textdegree
	\item 200\textdegree
	\item 330\textdegree
	\item 270\textdegree
\end{enumerate}


\Exercicio Expresa en grados las siguientes medidas angulares:

\begin{enumerate}[topsep=0pt]
	\item $ \frac{7\pi}{3}~rad $
	\item $ \frac{3\pi}{2}~rad $
	\item $ 4~rad$
	\item $ 4\pi~rad $
	\item $ \pi~rad $
	\item $ \frac{\pi}{2}~rad $
	\item $ \frac{\pi}{6}~rad $
\end{enumerate}

\section{Resolución de triángulos}

\Exercicio Resuelve los siguientes triángulos rectángulos:

\begin{enumerate}[topsep=0pt]
	\item \textbf{[C]} $\widehat{A} = 90º; b = 7~cm; c = 4~cm  $
	\item \textbf{[C]} $\widehat{A} = 90º; \widehat{C} = 60º; c = 7~cm  $
	\item \textbf{[C]} $\widehat{A} = 90º; a = 8~cm; b = 3~cm  $
	\item \textbf{[C]} $\widehat{A} = 90º; \widehat{C} = 30; c = 2~cm  $
	
	\item $\widehat{A} = 90º; a = 7~cm; b = 2~cm  $
	\item $\widehat{A} = 90º; \widehat{C} = 23; a = 6~m  $
	\item $\widehat{A} = 90º; \widehat{C} = 10; b = 10~m  $
	\item $\widehat{A} = 90º; a = 10~cm; b = 2~cm  $
	\item $\widehat{A} = 90º; \widehat{C} = 40; c = 8~dm  $
	\item $\widehat{A} = 90º; b = 5~cm; c = 2~cm  $
	\item $\widehat{A} = 90º; \widehat{C} = 10; b = 10~m  $
	\item $\widehat{A} = 90º; b = 9~cm; c = 4~cm  $
	\item $\widehat{A} = 90º; \widehat{C} = 5; b = 3~m  $
	\item $\widehat{A} = 90º; a = 7~cm; c = 1~cm  $

\end{enumerate}


\Exercicio Resuelve los siguientes triángulos no rectángulos:

\begin{enumerate}[topsep=0pt]
	\item \textbf{[C]} $\widehat{A} = 80º; \widehat{B} = 40º; a = 8~dm  $
	\item \textbf{[C]} $\widehat{A} = 80º;  a = 10~m; b = 5~m  $
	\item \textbf{[C]} $a = 10~cm; b = 15~cm; c = 20~~cm  $
	\item \textbf{[C]} $\widehat{A} = 75º; b = 8~mm; c = 12~mm  $

	\item $\widehat{A} = 75º; b = 8~mm; c = 12~mm  $
	\item $a=12~cm; b=15~cm; c = 20~cm $
	\item $\widehat{C}=80º; a=7cm; b=4~cm$
	\item $\widehat{A} = 50º; a=9~km; b=15~km$
	\item $\widehat{A} = 100; a=9~dm; b= 7~dm$
	\item $\widehat{A} = 35º; a=5~m; b=7~m$
	\item $\widehat{A} = 95º; \widehat{B} = 30º; a = 7~cm$
	\item $\widehat{A} = 40º; \widehat{B} = 70º; c = 10~mm$
	\item $\widehat{C} = 50º; a=16~m; b= 14~m$
	\item $a=8~cm; b=9~cm; c= 12~cm$
\end{enumerate}

\Exercicio Calcula en cada caso el resto de las razones trigonométricas:
\begin{enumerate}[topsep=0pt]
	\item \textbf{[C]} $sen(x) = 3/4$~~~~~~~$x \in (0,\pi/2)$
	\item \textbf{[C]} $cos(x) = 2/5$~~~~~~~$x \in (0,\pi/2)$
	\item \textbf{[C]} $tan(x) = 3$~~~~~~~$x \in (0,\pi/2)$
	\item $sen(x) = 0,8$~~~~~~~$x \in (0,\pi/2)$
	\item $cos(x)= \sqrt{5}/3$~~~~~~~$x \in (0,\pi/2)$
	\item $tan(x) = 8/15$~~~~~~~$x \in (0,\pi/2)$
	\item \textbf{[C]} $sen (x)=12/13$~~~~~~~$x \in (\pi/2,3\pi/2)$
	\item \textbf{[C]} $cos(x) = -0,96$~~~~~~~$x \in (0,\pi)$
	\item \textbf{[C]} $tan(x) = -8/15$~~~~~~~$x \in (\pi/2,\pi)$
	\item $cos(x)=-2/5$~~~~~~~$x \in (\pi, 2\pi)$
	\item $tan(x)=\sqrt{2}$~~~~~~~$x \in (\pi, 2\pi)$
\end{enumerate}


\Exercicio Indica en que cuadrante está cada uno de los siguientes ángulos y el signo de sus razones trigonométricas (seno, coseno y tangente):

\begin{enumerate}[topsep=0pt]
	\item \textbf{[C]} $ 100º $
	\item \textbf{[C]} $ 300º $
\end{enumerate}

\Exercicio Calcula las siguientes razones expresándolas en función del valor de ángulos del primer cuadrante

\begin{enumerate}[topsep=0pt]
	\item \textbf{[C]} $ sen(1380º) $
	\item \textbf{[C]} $ cos(1665º) $
	\item \textbf{[C]} $ tan(1920º) $
	\item \textbf{[C]} $ tan(690º) $
\end{enumerate}


\Exercicio Conociendo el valor de las razones trigonométricas de 30º, 45º, 60º; calcula las siguientes razones trigonométricas aplicando las fórmulas de las razones trigonométricas del ángulo suma y diferencia:

\begin{enumerate}[topsep=0pt]
	\item \textbf{[C]} $sen(75º)$
	\item \textbf{[C]} $cos(15º)$
	\item \textbf{[C]} $tan(195º)$
\end{enumerate}


\Exercicio Calcula empleando las formulas trigonométricas vistas:

\begin{enumerate}[topsep=0pt]
	\item \textbf{[C]} Sabiendo que $tan(\alpha) = -4$ y $\alpha \in (270º,360º)$:
		\begin{enumerate}[topsep=0pt, label=\arabic*)]
		\item  $tan(\alpha + 30º)$
		\item $cos(2\alpha)$
		\item $tan(\alpha - \frac{\pi}{3})$
		\item $tan(\frac{\alpha}{2})$
		\end{enumerate}

	\item \textbf{[C]} Sabiendo que $sen(\alpha) = \frac{1}{3}$ y $\alpha \in (0º,90º)$:
		\begin{enumerate}[topsep=0pt, label=\arabic*)]
		\item $sen(\alpha + 45º)$
		\item $cos(2\alpha)$
		\item $tan(\alpha + \frac{\pi}{6})$
		\item $cos(\frac{\alpha}{2})$
		\end{enumerate}
	\item \textbf{[C]} Sabiendo que $sen(\alpha) = \frac{3}{5}$, que $\alpha \in (90,180º)$, que $cos(\beta) = \frac{5}{13}$ y que $ \beta \in (0º,90º)$:
		\begin{enumerate}[topsep=0pt, label=\arabic*)]
		\item $sen(\alpha - \beta)$
		\item $cos(3\beta)$
		\item $cos(\frac{\alpha}{2})$
		\end{enumerate}
\end{enumerate}


\Exercicio Transforma en productos y calcula:

\begin{enumerate}[topsep=0pt]
	\item \textbf{[C]} $cos(15º) - cos(75º)$
	\item \textbf{[C]} $sen(30º) + sen(330º)$
\end{enumerate}


\Exercicio Transforma en sumas o diferencias y calcula:

\begin{enumerate}[topsep=0pt]
	\item \textbf{[C]} $cos(225º)sen(15º)$
	\item \textbf{[C]} $sen(285º)sen(75º)$
\end{enumerate}


\Exercicio Demuestra que se cumplen las siguientes expresiones usando las fórmulas trigonométricas vistas:

\begin{enumerate}[topsep=0pt]
	\item \textbf{[C]} $ \frac{cos(\alpha - \beta)}{cos(\alpha + \beta)} = \frac{1 + tan(\alpha) \cdot tan(\beta)}{1-tan(\alpha)tan(\beta)} $
	\item \textbf{[C]} $ cos(x + \frac{\pi}{3}) - cos(x + \frac{2\pi}{3}) = cos(x)$
	\item \textbf{[C]} $ \frac{2 sen(x) - sen(2x)}{2 sen(x) - sen(2x)} = \frac{1 - cos(x)}{1 + cos(x)}$
	\item \textbf{[C]} $ \frac{sen(2\alpha)}{1 + cos(2\alpha)} = tan(\alpha)$
\end{enumerate}


\Exercicio Simplifica:

\begin{enumerate}[topsep=0pt]
	\item \textbf{[C]} $\frac{sen(40º) + sen(50º)}{cos(40º) - cos(50º)}$
	\item \textbf{[C]} $\frac{2sen(\alpha - \beta)}{sen(2\alpha)}$
	\item \textbf{[C]} $2tan(\alpha) - sen^2(\frac{\alpha}{2}) + sen(\alpha)$
\end{enumerate}


\Exercicio Resuelve las siguientes ecauciones trigonométricas:

\begin{enumerate}[topsep=0pt]
	\item \textbf{[C]}
	\item \textbf{[C]}
	\item \textbf{[C]}
\end{enumerate}

