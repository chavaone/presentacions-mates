\chapter{Límites de funciones. Continuidad y asíntotas}
\setcounter{exercicio}{0}

\section {Concepto de límite. Límites laterales}
\Exercicio Calcula los siguientes límites analizando la gráfica A.7 del anexo de funciones:
\begin{enumerate}[topsep=0pt]
	\item \textbf{[C]} $ \lim_{x \to -8^-} f(x) $
	\item \textbf{[C]} $ \lim_{x \to -8^+} f(x) $
	\item \textbf{[C]} $ \lim_{x \to 11^-} f(x) $
	\item \textbf{[C]} $ \lim_{x \to 11^+} f(x) $
	\item \textbf{[C]} $ \lim_{x \to 5^-} f(x) $
	\item \textbf{[C]} $ \lim_{x \to 5^+} f(x) $
\end{enumerate}


\Exercicio Calcula los límites cuando x tiende a 3 de las siguientes funciones:
\begin{enumerate}[topsep=0pt]
	\item \textbf{[C]} $ f(x) = 3x + 4 $
	\item \textbf{[C]} $ f(x) = \frac{1}{x} $
	\item \textbf{[C]} $ f(x) = \left\{ \begin{array}{lcc}
						x - 2 &   si  & x < 3 \\
						1 &  si & x \ge 3
						\end{array}
						\right.$
	\item \textbf{[C]} $ f(x) = \left\{ \begin{array}{lcc}
						x^2 -6x + 9 &   si  & x < 3 \\
						x-6 &  si & x \ge 3
						\end{array}
						\right.$
\end{enumerate}


\section{Límites en el infinito y en 0}

\Exercicio Calcula los siguientes límites empleando las propiedades de las operaciones con el 0 y el $ \infty $:
\begin{enumerate}[topsep=0pt]
	\item \textbf{[C]} $ \lim_{x \to 0} 3 + \frac{1}{2x}$
	\item \textbf{[C]} $ \lim_{x \to -\infty}  x^2+3x$
	\item \textbf{[C]} $ \lim_{x \to \infty} ln(x) + \sqrt{x+3} $
	\item \textbf{[C]} $ \lim_{x \to \infty} \sqrt{2x} - \sqrt{x+1}  $
	\item \textbf{[C]} $ \lim_{x \to \infty} (2x + 3) \cdot ln(x) $
	\item \textbf{[C]} $ \lim_{x \to \infty} \frac{2^-x}{x^2 + 3}$
	\item \textbf{[C]} $ \lim_{x \to \infty} \frac{ln(x)}{e^{-2x+3}}$
	\item \textbf{[C]} $ \lim_{x \to \infty} \frac{x^3-2}{x^2+3x}$
	\item \textbf{[C]} $ \lim_{x \to -1} \frac{x^2-1}{x+1}$
	\item \textbf{[C]} $ \lim_{x \to \infty} (\frac{2}{x})^{x+3}$
	\item \textbf{[C]} $ \lim_{x \to 0} (1-x)^{\frac{1}{x}}$
\end{enumerate}

\section{Indeterminaciones}


\Exercicio Resuelve los siguientes límites aplicando lo visto sobre indeterminaciones de tipo $ \frac{\infty}{\infty} $:
\begin{enumerate}[topsep=0pt]
	\item \textbf{[C]} $ \lim_{x \to \infty} \frac{3x^2-2x+1}{-x^3+2x^2+x-3} $
	\item \textbf{[C]} $ \lim_{x \to \infty} \frac{x^2-3x+2}{2x^2-x+5} $
	\item \textbf{[C]} $ \lim_{x \to \infty} \frac{4x^3-x^2+2x+5}{-2x^2+7x-3} $
	\item \textbf{[C]} $ \lim_{x \to \infty} \frac{-x^4-5x^2+3}{2x^3+x} $
	\item \textbf{[C]} $ \lim_{x \to \infty} \frac{x-4}{\sqrt{x^2-1}} $
	\item \textbf{[C]} $ \lim_{x \to \infty} \frac{x^2-4}{\sqrt[3]{x^7-1}} $
	\item \textbf{[C]} $ \lim_{x \to \infty} \frac{e^x}{x^2 +3x} $
\end{enumerate}


\Exercicio Resuelve los siguientes límites aplicando lo visto sobre indeterminaciones de tipo $ \frac{0}{0} $:
\begin{enumerate}[topsep=0pt]
	\item \textbf{[C]} $ \lim_{x \to 1} \frac{x^3-x^2}{x^2+3x-4} $
	\item \textbf{[C]} $ \lim_{x \to 0} \frac{x^2+3x}{2x^2-x}  $
	\item \textbf{[C]} $ \lim_{x \to -1} \frac{x^3+x^2+x+1}{x^2-x}  $
	\item \textbf{[C]} $ \lim_{x \to 1} \frac{x-1}{\sqrt{x}-1}  $
	\item \textbf{[C]} $ \lim_{x \to 0} \frac{ \sqrt{1+x}-\sqrt{1-x}}{x} $
\end{enumerate}


\Exercicio Resuelve los siguientes límites aplicando lo visto sobre indeterminaciones de tipo $ \infty \cdot 0 $:
\begin{enumerate}[topsep=0pt]
	\item \textbf{[C]} $ \lim_{x \to \infty } e^{-x} \cdot (x^2 + x + 1) $
	\item \textbf{[C]} $ \lim_{x \to 1} \frac{1}{\sqrt{1} -1} \cdot (x-1) $
\end{enumerate}


\Exercicio Resuelve los siguientes límites aplicando lo visto sobre indeterminaciones de tipo $ \frac{\infty}{\infty} $:
\begin{enumerate}[topsep=0pt]
	\item \textbf{[C]} $ \lim_{x \to 3} (\frac{4}{x-3} - \frac{3x}{x^2-x-6}) $
	\item \textbf{[C]} $ \lim_{x \to \infty} ( \frac{2x^4-x^2+1}{x^3} - \frac{2x^3+x^2-x+4}{3x^2} ) $
	\item \textbf{[C]} $ \lim_{x \to \infty} (\sqrt{4x^2-2x+6} - 2x+1) $
	\item \textbf{[C]} $ \lim_{x \to 2} (\frac{1}{\sqrt{x-2}} - \frac{3}{\sqrt{x^2-4}}) $
\end{enumerate}


\Exercicio Resuelve los siguientes límites aplicando lo visto sobre indeterminaciones de tipo $ \frac{\infty}{\infty} $:
\begin{enumerate}[topsep=0pt]
	\item \textbf{[C]} $ \lim_{x \to \infty} (1 - \frac{2}{x})^{x+1} $
	\item \textbf{[C]} $ \lim_{x \to \infty} (\frac{x^2-2x+1}{x^2-4})^{\frac{x^2}{x-2}} $
	\item \textbf{[C]} $ \lim_{x \to \infty} (\frac{x^2-1}{x^2+1})^{x} $
	\item \textbf{[C]} $ \lim_{x \to \infty} (\frac{x}{2x-1})^{\frac{1}{x+1}} $
\end{enumerate}

\section{Continuidad de funciones}

\Exercicio Indica si las siguientes funciones son continuas en el punto $ x_0 = 3 $:
\begin{enumerate}[topsep=0pt]
\item \textbf{[C]} $f(x) = x^2-3x$
	\item \textbf{[C]} $g(x) = \frac{x^2-5x+6}{x-3} $
	\item \textbf{[C]} $t(x) = \frac{x}{x^2-9}$
	\item \textbf{[C]} $f(x) = \left\{ \begin{array}{lcc}
						x - 2 &   si  & x < 3 \\
						1 &  si & x \ge 3
						\end{array}
						\right.$
	\item \textbf{[C]} $f(x) = \left\{ \begin{array}{lcc}
						x^2 - 4 &   si  & x < 3 \\
						3 &  si & x \ge 3
						\end{array}
						\right.$
\end{enumerate}


\Exercicio Indica el tipo de discontinuidad que presentaban en las funciones del ejercicio anterior.

\Exercicio Estudia la continuidad de las siguientes funciones:
\begin{enumerate}[topsep=0pt]
	\item \textbf{[C]} $f(x) = 3x + 4$
	\item \textbf{[C]} $f(x) = \left\{ \begin{array}{ll}
						2x+6      &  x < 1 \\
						x-7       &  x \ge 1
						\end{array}
						\right.$
	\item \textbf{[C]} $ f(x) = \left\{ \begin{array}{ll}
			             2x+3           &    x \le 1 \\
			             x^2+2x+2       &    1 < x < 2 \\
						 -3             &    x \ge 2
			             \end{array}
			   			\right.$
	\item \textbf{[C]} $ f(x) = \left\{ \begin{array}{ll}
						 \frac{2x+6}{x}      &    x < 1 \\
						 x+7                 &  x \ge 1
						 \end{array}
						 \right.$
\end{enumerate}


\Exercicio Calcula el valor del parámetro $a$ para que las siguientes funciones sean continuas:
\begin{enumerate}[topsep=0pt]
	\item \textbf{[C]} $ f(x) = \left\{ \begin{array}{ll}
							 2x+a      &    x \le 1 \\
							 x^2 - ax + 2       &  x < 1
							 \end{array}
		 				\right.$
	\item \textbf{[C]} $ f(x) = \left\{ \begin{array}{ll}
							 e^{ax}      &    x \le 0 \\
							 x+2a        &    x < 0
							 \end{array}
						 \right.$
\end{enumerate}

\section{Asíntotas}

\Exercicio Estudia la presencia de asíntotas en las siguientes funciones:
\begin{enumerate}[topsep=0pt]
	\item \textbf{[C]} $f(x) = \frac{x-2}{x^2-4} $
	\item \textbf{[C]} $f(x) = \frac{2x^2-1}{x+2} $
	\item \textbf{[C]} $f(x) = \frac{3x^2+x+1}{x^2-3x-4}$
	\item \textbf{[C]} $f(x) = \frac{x^2-3x}{x^2-x-6}$
\end{enumerate}
