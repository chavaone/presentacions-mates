\chapter{Derivada}
\setcounter{exercicio}{0}

\section{Derivada de una función en un punto}

\Exercicio Calcula las siguientes derivadas usando la definición:

\begin{enumerate}[topsep=0pt]
	\item \textbf{[C]} Siendo $f(x) = 3x - 4$, $f'(2)$
	\item \textbf{[C]} Siendo $g(x) = x^2 - 4$, $g'(5)$
	\item Siendo $f(x) = \frac{1}{x}$, $f'(2)$.
	\item Siendo $f(x) = 3x^2-4x$, $f'(2)$.
\end{enumerate}

\section{Derivabilidad}

\Exercicio Estudia la continuidad y derivabilidad de las siguientes funciones:
\begin{enumerate}[topsep=0pt]
	\item \textbf{[C]} \textit{[Jn10]} $ f(x) = \left\{ \begin{array}{lc}
						x^2+1 & x \leq 1 \\
						-x+3  & x > 1
						\end{array}
						\right.$
	\item \textbf{[C]} $ t(x) = |x-5| $
	\item \textbf{[C]} $ f(x) = \left\{ \begin{array}{lc}
						\frac{x^2-9}{x-3} & x \neq 3 \\
						6                 & x = 1
						\end{array}
						\right.$
	\item \textit{[Se07]} $ f(x) = \left\{ \begin{array}{lc}
			0 & x \leq \sqrt{2} \\
			-x^2+2 & x \leq \sqrt{2}
			\end{array}
			\right.$
\end{enumerate}


\Exercicio Comprueba si existe algún valor de $a$ (y de $b$) para el cual sean derivables las siguientes funciones en los puntos indicados.
\begin{enumerate}[topsep=0pt]
	\item \textbf{[C]} \textit{[Ju16]} $ f(x) = \left\{ \begin{array}{lc}
						 ax+2 & x < 1 \\
						 3(x-2)^2  & x \ge 1
						 \end{array}
	 					\right.$
	 					en $x=1$.
	\item \textbf{[C]} $ f(x) = \left\{ \begin{array}{lc}
						 a-x^2 & x < 1 \\
						 \frac{2}{ax}  & x \ge 1
						 \end{array}
	 					\right.$
	 					en $x=1$.
	 \item $ f(x) = \left\{ \begin{array}{lc}
		 	a & x < -2 \\
		 	x^2-bx+1  & x \ge -2
			\end{array}
			\right.$
		 	en $x=1$.
	 \item $ f(x) = \left\{ \begin{array}{lc}
			2ax +b & x < 2 \\
			x^3+ax+2  & x \ge 2
			\end{array}
			\right.$
			en $\mathbb{R}$.

\end{enumerate}

\section{Recta tangente y normal}

\Exercicio Calcula la ecuación de la recta tangente y de la recta normal a las siguientes funciones en los puntos indicados:
\begin{enumerate}[topsep=0pt]
	\item \textbf{[C]} $ f(x) = 3x^2 + 4x $ en  $x_0 = 3$
	\item \textbf{[C]} $ t(x) = \frac{1}{x} $ en $x_0 = 1$
	\item $s(x) = x^2 -10$ en $x_0 = 0$
	\item $m(x) = \sqrt{3x + 2}$ en $x_0 = 10$
\end{enumerate}

\Exercicio Determina los puntos donde la recta tangente a las funciones dadas cumple las propiedades indicadas. Calcula también dicha recta tangente.
\begin{enumerate}[topsep=0pt]
	\item \textbf{[C]} La tangente de $f(x) = x^2+3x$ y la recta $r \equiv 2x+y -3 = 0$ son paralelas.
	\item \textbf{[C]} La tangente de $f(x) = \frac{x^3}{3} - x^2 -3x +1$ forma un ángulo de 135º con el sentido positivo del eje de abscisas.
	\item La tangente de la función $f(x) = x^4 - 2x$ y la recta $r \equiv 3x - 6y +10 = 0$ son paralelas.
	\item La tangente de $f(x) = \frac{1}{x^2 + 4}$ y la recta $r \equiv 2x - 4y +10 = 0$ son paralelas.
	\item La tangente de $f(x) = x^3 + 10x$ forma un ángulo de 20º con el sentido positivo del eje de abscisas.
\end{enumerate}


\section{Función derivada y cálculo de derivadas}

\Exercicio Calcula la función derivada primera y segunda de las siguientes funciones usando la definición de derivada:
\begin{enumerate}[topsep=0pt]
	\item \textbf{[C]} $f(x) = 3x$
	\item \textbf{[C]} $g(x) = x^2 + 3$
\end{enumerate}


\Exercicio Calcula las derivadas de las siguientes funciones polinómicas:

\begin{enumerate}[topsep=0pt]
	\item \textbf{[C]} $ f(x) = 3$
	\item \textbf{[C]} $ f(x) = x^3 $
	\item \textbf{[C]} $ f(x) = x^3 + 2x $
	\item \textbf{[C]} $ f(x) = x^2 + 3x + 4 $
	\item \textbf{[C]} $ f(x) = 2x^4 + 5x^2 $
	\item $ f(x) =  3x^5 +4x^4 -2x^3 -5x^2 +6x +8$
	\item $ f(x) =  \frac{3x^4}{5} + \frac{9x^2}{2} + 5x -15$
	\item $ f(x) =  \frac{x^2-5x^4+12x^3}{2}$
\end{enumerate}


\Exercicio Calcula las derivadas de las siguientes funciones elementales:

\begin{enumerate}[topsep=0pt]
	\item \textbf{[C]} $ f(x) = 2^x $
	\item \textbf{[C]} $ f(x) = ln(x) $
	\item \textbf{[C]} $ f(x) = sen(x) + cos(x)$
	\item \textbf{[C]} $ f(x) = \sqrt{x}$
\end{enumerate}


\Exercicio Calcula las derivadas de los siguientes productos y cocientes de funciones:

\begin{enumerate}[topsep=0pt]
	\item \textbf{[C]} $ f(x) = 2x \cdot 2^x $
	\item \textbf{[C]} $ f(x) = sen(x) \cdot \sqrt{x} $
	\item \textbf{[C]} $ f(x) = \frac{sen(x)}{2x} $
	\item \textbf{[C]} $ f(x) = \frac{e^x}{3x^2}$
\end{enumerate}


\Exercicio Calcula las siguientes derivadas:

\begin{enumerate}[label=a\alph*.,topsep=0pt]
	\item \textbf{[C]} $ f(x) = sen(x^3 -2x) $
	\item \textbf{[C]} $ f(x) = ln(x^2+4) $
	\item \textbf{[C]} $ f(x) = cos((x+3)^2) $
	\item \textbf{[C]} $ f(x) = 2^{x^2+3x} $
	\item \textbf{[C]} $ f(x) = ln(3x \cdot 3^x) $
	\item \textbf{[C]} $ f(x) = \sqrt{sen(x^2 +3)} $
	\item \textbf{[C]} $ f(x) =  \frac{1}{cos^2(x^2 + 3)}$
	\item \textbf{[C]} $ f(x) =  (\frac{x^2+3}{x-4})^3$
	\item \textbf{[C]} $ f(x) =  \frac{sen((x+3)^3)}{2x^2}$
	\item \textbf{[C]} $ f(x) =  \sqrt[3]{\frac{x^2+3}{sen(x)}}$
	\item \textbf{[C]} $ f(x) =  ln(\sqrt{\frac{x^2+3}{cos(x)}})$
	\item \textbf{[C]} $ f(x) = tan(x^2 + \sqrt{3x})$
	\item $ f(x) =  (x^2+1)^3$
	\item $ f(x) =  (x^4 +x^2 -3)^{-2}$
	\item $ f(x) =  x^{1/3}$
	\item $ f(x) =  (x^2+3x-5)^{1/2}$
	\item $ f(x) =  \sqrt{x^2-x}$
	\item $ f(x) =  \sqrt[3]{x^5-4x}$
	\item $ f(x) =  7^{x^2+5x}$
	\item $ f(x) =  e^{x^2}$
	\item $ f(x) =  e^{x^2+2x-1}$
	\item $ f(x) =  ln(x^2-x)$
	\item $ f(x) =  ln(x^3)$
	\item $ f(x) =  2 \cdot ln(3x+5)$
	\item $ f(x) =  \frac{(5-x)^2}{3x-1}$
	\item $ f(x) =  \frac{x}{(x^2+1)^2}$
	\item $ f(x) =  \frac{3x}{ln(x)}$
\end{enumerate}
\begin{enumerate}[label=b\alph*.,topsep=0pt]
	\item $ f(x) =  x \cdot \sqrt{x^2-x}$
	\item $ f(x) =  cos((3x^2 -x)^3)$
	\item $f(x) = sen^3(x^2-4)$
	\item $f(x) = cos^5(x^5+1)$
	\item $f(x) = (ln(6x+4))^2$
	\item $f(x) = x^3 \cdot sen(x^3)$
	\item $f(x) = e^{-x} \cdot sen(2x^5)$
	\item $f(x) = tan(2x^2+x)$
	\item $f(x) = tan(sen(x))$
	\item $f(x) = \sqrt{e^{5x-1}}$
	\item $f(x) = ln(\frac{3x^2-1}{x+5})$
	\item $f(x) = \sqrt{x+1}{x-1}$
	\item $f(x) = \frac{1+sen(x)}{1-sen(x)}$
	\item $f(x) = \sqrt{\frac{1-e^{-x}}{1+e^{-x}}}$
	\item $f(x) = (x^2-1)\cdot ln(2x)$
	\item $f(x) = \frac{1}{2} ln(\frac{1+sen(x)}{1-sen(x)})$
	\item $f(x) = sen^3(x^2)$
	\item $f(x) = ln(\sqrt{\frac{1-e^-x}{1+e^-x}})$
	\item $f(x) = e^{3 \cdot cos(x)}$
\end{enumerate}
