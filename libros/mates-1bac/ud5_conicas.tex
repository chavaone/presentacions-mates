\chapter{Cónicas}
\setcounter{exercicio}{0}

\Exercicio Calcula la ecuación de la circunferencia en cada caso:

\begin{enumerate}[topsep=0pt]
	\item \textbf{[C]} Su centro y radio son $C(-3,1)$ y $r = \frac{2}{3}$
	\item \textbf{[C]} Su centro y radio son $C(1,3)$ y $r=5$
	\item \textbf{[C]} Su centro es C(3,1) y pasa por el punto P(5,-2)
	\item \textbf{[C]} Pasa por los puntos P(1,5), Q(3,7) y R(-1,9)
\end{enumerate}


\Exercicio Calcula el centro y radio de las siguientes circunferencias:

\begin{enumerate}[topsep=0pt]
	\item \textbf{[C]} $x^2+y^2 -8x -2y = -13$
	\item \textbf{[C]} $16x^2 + 16y^2 -16x -32y = -11 $
	\item \textbf{[C]} $4x^2 + 4y^2 -12y = -5$
\end{enumerate}


\Exercicio Calcula la posición relativa de los siguientes pares de objectos geométricos:

\begin{enumerate}[topsep=0pt]
	\item \textbf{[C]} La circunferencia $x^2+2x+y^2-4y=4$ y el punto $P(3,2)$
	\item \textbf{[C]} La circunferencia $x^2-4x+y^2 - 6y +12 = 0$ y la recta $ 3x - 4y + 1 = 0$
	\item \textbf{[C]} La circunferencia $x^2 -2x + y^2 -8y +16 = 0$ y la circunferencia $x^2 + 8x + y^2 - 2y + 13=0$
\end{enumerate}


\Exercicio Calcula la ecuación de la elipse a partir de los siguientes datos:

\begin{enumerate}[topsep=0pt]
	\item \textbf{[C]} Focos en los puntos $F'(-3,0)$ y $F(3,0)$ y vértices dos vértices $A(6,0)$ y $A'(-6,0)$
\end{enumerate}

