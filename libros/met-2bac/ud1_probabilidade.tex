\chapter{Probabilidade Condicional}
\setcounter{exercicio}{0}

\section{Sucesos}

\Exercicio  Clasifica os seguintes experimentos como deterministas ou aleatorios:

\begin{enumerate}[topsep=0pt,itemsep=0pt]
	\item Lanzar unha moeda ao aire e anotar se sae cara ou cruz.
	\item Comprobar o tempo que tarda un obxecto en recorrer unha distancia coñecidas velocidade e condicións do entorno.
	\item O tempo metereolóxico que fará mañá.
\end{enumerate}


\Exercicio  No experimento de lanzar un dado de 12 caras e observar o seu resultado	calcula:

\begin{enumerate}[topsep=0pt,itemsep=0pt]
	\item Os sucesos elementales.
	\item O espazo mostral.
	\item O suceso $A =$ “sacar un múltiplo de 3”.
	\item O suceso $B =$ “sacar un números par”.
	\item Un suceso imposible.
\end{enumerate}

\section{Álxebra de sucesos}

\Exercicio  Dado o experimento lanzar un dado de 6 caras e observar o resultado e sexan $A$, $B$ y $C$ os sucesos sacar un número par, sacar un número menor que 3 y sacar un múltiplo de 3, respectivamente;  expresa empregando notación e calcula os sucesos elementais que forman:

\begin{enumerate}[topsep=0pt,itemsep=0pt]
	\item Ocorre $A$ e $B$ pero non $C$.
	\item Os tres sucesos ocorren simultáneamente.
	\item Ocorren $A$ ou $B$, pero non $C$
	\item Ocorre algún dos tres sucesos.
	\item Ningún dos tres sucesos ocorre.
\end{enumerate}


\section{Regra de Laplace}

\Exercicio  Dunha baralla española de 40 cartas extráese unha. Calcula as seguintes probabilidades:

\begin{enumerate}[topsep=0pt,itemsep=0pt]
	\item Que sexa un rei.
	\item Que sexa de copas
	\item Que no sexa figura (sota, cabalo ou rei).
	\item Que sexa o 7 de espadas.
\end{enumerate}


\Exercicio Dunha urna que contén 10 bólas numeradas do 1 ao 10 extráese unha bóla. Consideremos os sucesos A = obter número par, B = obter un número maior que 7 e C = obter un múltiplo de tres. Calcula as probabilidades dos sucesos:

\begin{enumerate}[topsep=0pt,itemsep=0pt]
	\item $A$, $B$ e $C$
	\item $A \cap B$
	\item $A \cup B$ e $A \cup B \cup C$
	\item $A - B $
\end{enumerate}


\Exercicio Dado o experimento lanzar dous dados de 6 caras, calcula as probabilidades dos seguintes sucesos:

\begin{enumerate}[topsep=0pt,itemsep=0pt]
	\item Sacar o mesmo número nos dous dados.
	\item Que os números sumen 7.
\end{enumerate}


\section{Propiedades da probabilidade}

\Exercicio Sexan A e B dous sucesos incompatibles dun experimento aleatorio tales que $P(A)= 0,2$ e $P(A\cup B)=0,6$. Calcula $P(B)$.

\Exercicio Considéranse os sucesos $A$ e $B$ asociados a un experimento aleatorio con $P(A)= 0,7$; $P(B)=0,6$ e $P(A\cap B) = 0,4$. Calcula $ P(A\cup B)$, $ P(\overline{A} \cup \overline{B})$ e $ P(A - B)$.

\Exercicio Sexan os sucesos A e B tales que $P(A) = 3/8$, $P(B)= 1/2$ e $P(A \cap B) = 1/4$. Calcula $P(A \cup B)$, $P(\overline{A})$, $P(\overline{A} \cup \overline{B})$ e $P(\overline{A} \cap \overline{B})$ e e $ P(A - B)$.

\section{Probabilidade condicional}

\Exercicio Nunha caixa hai pinzas grandes e pequenas de madeira e de plástico según se reflexa na táboa. Se se elixe unha ao azar, calcule a probabilidade de:

\begin{enumerate}[topsep=0pt,itemsep=0pt]
	\item Que sexa grande.
	\item Que sexa grande e de plástico.
	\item Que sexa grande sabendo que é de plástico.
\end{enumerate}

\begin{center}
	\begin{tabular}{l | c | c}
		& Grandes & Pequnas \\
		Madera   & 10      & 19      \\
		Plástico & 18      & 23     
	\end{tabular}
\end{center}


\Exercicio Que probabilidades nos dan e qué probabilidades nos piden no seguinte problema?:

$H = \text{ser home} \\
 M = \text{ser muller} \\
 A = \text{adquirir un produto}$

\begin{quote}
	Segundo certo estudo dun departamento de vendas, o 30\% dos seus clientes son homes, o 25\% dos seus clientes adquiren algún produto e o 40\% dos que adquiren algún produto son mulleres. Que porcentaxe dos seus clientes son mulleres e adquiren algún produto do departamento de electrónica?[...]
\end{quote}

\Exercicio Que probabilidades nos dan e qué probabilidades nos piden no seguinte problema?:

$D = \text{estar defectuoso} \\
 I = \text{pasar a inspección}$
\begin{quote}
	Cando os motores chegan ó final dunha cadea de produción, escóllense os que deben pasar unha inspección. Prodúcense un 10\% de motores defectuosos, e o 60\% de tódolos motores defectuosos e o 20\% dos bós pasan unha inspección. ¿Probabilidade de que un motor sexa defectuoso e pase a inspección e de que un motor sexa bo e pase a inspección?[...]
\end{quote}

\Exercicio  Nunha enquisa realizada en A Coruña determinou que o 40\% dos enquisados lee o xornal A Voz de Galicia, o 15\% lee o Nós Diario e o 3\% lee ambos xornais. Selecionado un lector ao azar do xornal Nós Diario, calcular a probabilidade de que lea tamén La Voz de Galicia.

\section{Regra do produto. Independencia de sucesos}

\Exercicio Dado o experimento, tirar un dado tres veces e calcular a probabilidade de sacar un número par a primeira vez, sacar 6 a segunda vez e sacar un múltiplo de 3 a terceira.

\Exercicio Temos unha bolsa con 3 bólas blancas e 5 negras. Calcular a probabilidade de que ó sacar dúas bólas a primeira sea branca e a segunda negra.

\Exercicio  (MACS - Xuño 2003) Sexan $A$ e $B$ dous sucesos tales que $P(A)=0,6$ e $P(B)=0,3$. Se $P(A/B)=0,1$ calcúlese $P(A \cup B)$ e $P( \overline{B}/ A )$.

\Exercicio  (Mat - Xuño 2017) Nun experimento aleatorio, sexan A e B dous sucesos con $P(\overline{A})=0'4$; $P(B)=0'7$ . Se $A$ e $B$ son independentes, calcula $P(A \cup B)$ e $P(A-B)$

\Exercicio  (MACS – Setembro 2014) Sábese que $P(B/A)=0'7$ , $P(A/B)=0'4$ e $P(A)=0'2$. Calcula $P( A \cap B)$, $P( B)$ e $P(A \cup \overline{B})$. Xustifica se son independentes ou non os sucesos $A$ e $B$.

\section{Probabilidades totais}

\Exercicio  (MACS - Xuño 2017) Segundo certo estudo dun departamento de vendas, o 30\% dos seus clientes son homes, o 25\% dos seus clientes  adquiren  algún produto e o 40\% dos que  adquiren  algún produto son mulleres. ¿Que porcentaxe dos seus clientes son mulleres e adquiren algún produto do departamento de electrónica? 

\Exercicio  (MACS - Xuño 2001)Cando os motores chegan ó final dunha cadea de produción, escóllense os que deben pasar unha inspección. Prodúcense un 10\% de motores defectuosos, e o 60\% de tódolos motores defectuosos e o 20\% dos bós pasan unha inspección. ¿Probabilidade de que un motor sexa defectuoso e pase a inspección e de que un motor sexa bo e pase a inspección?


\Exercicio  (MACS - Xuño 2017) Un artigo distribuído en tres marcas distintas A, B e C; véndese nun supermercado. Obsérvase que o 30\% das vendas diarias do artigo son da marca A, o 50\% son da marca B e o resto son da marca C. Sábese ademais que o 60\% das vendas da marca A realízase pola mañá, o 55\% das vendas da marca B pola tarde e o 40\% da marca C véndese pola mañá. Calcula a porcentaxe de vendas do artigo efectuadas pola mañá.

\Exercicio  (MACS - Xuño 2010) Un estudo sociolóxico afirma que 3 de cada 10 persoas dunha determinada poboación son obesas, das cales o 60\% segue unha dieta. Por outra parte, o 63\% da poboación non é obesa e non segue unha dieta. Que porcentaxe da poboación segue unha dieta?

\Exercicio  (MACS - Xuño 2005) O cadro de persoal duns grandes almacéns está formado por 200 homes e 300 mulleres. A cuarta parte dos homes e a terceira parte das mulleres só traballan no turno da mañá. Elexido un dos empregados ó chou, cal é a probabilidade de que sexa home ou só traballe no turno da mañá?

\Exercicio  (MACS - Setembro 2007)Nunha cidade, o 55\% da poboación en idade laboral son homes; deles, un 12\% está no paro. Entre as mulleres a porcentaxe de paro é do 23\%. Se nesta cidade se elixe ao chou unha persoa en idade laboral:

\begin{enumerate}[topsep=0pt,itemsep=0pt]
	\item Cal é a probabilidade de que sexa home e non estea no paro?
	\item Cal é a probabilidade de que sexa muller e estea no paro?
	\item Calcular a porcentaxe de paro nesa cidade
\end{enumerate}

   
\Exercicio  (MACS - Xuño 2007) Nunha cidade na que hai dobre número de homes que de mulleres declárase unha epidemia. Un 4\% dos habitantes son homes e están enfermos, mentres que un 3\% son mulleres e están enfermas. Elíxese ao chou un habitante da cidade, calcular:

\begin{enumerate}[topsep=0pt,itemsep=0pt,itemsep=0pt]
	\item Probabilidade de que sexa home.
	\item Se é home, a probabilidade de que estea enfermo.
	\item A probabilidade de que sexa muller ou estea sa.
\end{enumerate}


\Exercicio  (MACS - Xuño 2008) Nun mercado de valores cotizan un total de 60 empresas, das que 15 son do sector bancario, 35 son industriais e 10 son do sector tecnolóxico. A probabilidade de que un banco dos que cotizan no mercado se declare en creba é 0,01, a probabilidade de que se declare en creba unha empresa industrial é 0,02 e de que o faga unha empresa tecnolóxica é 0,1. ¿Cal é a probabilidade de que se produza unha creba nunha empresa do citado mercado de valores?

\Exercicio (MACS - Xuño 2018) Nunha empresa, o 20\% dos traballadores son maiores de 30 anos, o 8\% desempeña algún posto directivo e o 6\% é maior de 30 anos e desempeña algún posto directivo.

\begin{enumerate}[topsep=0pt,itemsep=0pt]
	\item Que porcentaxe dos traballadores ten máis de 30 anos e non desempeña ningún cargo directivo?
	\item Que porcentaxe dos traballadores non é directivo nin maior de 30 anos?
\end{enumerate}


\Exercicio  Realízase un estudo para determinar se os fogares dunha pequena cidade se subscribirían a un servizo de TV. Os fogares clasifícanse de acordo ao seu nivel de renda.
\begin{center}
	\begin{tabular}{llll}
		& R. Baixa & R. Media & R. Alta \\
		Subscribiríanse      & 0,05        & 0,15        & 0,1        \\
		Non se subscribirían & 0,15        & 0,47        & 0,08      
	\end{tabular}
\end{center}

\begin{enumerate}[topsep=0pt,itemsep=0pt]
	\item Se o fogar subscribe o servizo, ¿cal é a probabilidade de que sexa de renda alta?
	\item Son renda e posible subscrición á televisión por cable independentes? Xustificar a resposta.
	\item Calcula a probabilidade de que un fogar seleccionado ao chou pertenza polo menos a unha destas categorías: “renda media” ou “desexan subscribirse”.
	
\end{enumerate}


\section{Teorema de Bayes}

\Exercicio Unha empresa quere comercializar unha ferramenta eléctrica para a construción e polo tanto é probada por 3 de cada 5 traballadores do sector. Dos que a probaron, o 70\% dá unha opinión favorable, o 5\% dá unha opinión desfavorable e o resto opina que lle é indiferente. Dos que non probaron a ferramenta, o 60\% dá unha opinión favorable, o 30\% opina que lle é indiferente e o resto dá unha opinión desfavorable. Sábese que a empresa comercializará a ferramenta se ao menos o 65\% dos traballadores do sector dá unha opinión favorable.

\begin{enumerate}[topsep=0pt,itemsep=0pt]
	\item Se un traballador elexido ao chou dá unha opinión desfavorable, ¿cal é a probabilidade de que probara a ferramenta?
	\item Que porcentaxe de traballadores dá unha opinión favorable? ¿Comercializará a empresa a ferramenta? Razoa a resposta.
	\item Calcula a porcentaxe de traballadores que proba a ferramenta e opina que lle é indiferente.
\end{enumerate}


\Exercicio  (MACS - Xuño 2011) Quérese facer un estudo sobre a situación laboral dos traballadores en tres sectores da economía que denotaremos por B1, B2 e B3. A metade dos traballadores pertencen ao primeiro sector B1, e o resto repártense en partes iguais entre os outros dous sectores B2 e B3. O 8\% dos do sector B1, o 4\% dos do sector B2 e o 6\% dos do sector B3 están no paro.

\begin{enumerate}[topsep=0pt,itemsep=0pt]
	\item Calcula a porcentaxe de paro entre os traballadores de dito estudo.
	\item Que porcentaxe dos que teñen traballo pertencen ao terceiro sector B3?
\end{enumerate}
